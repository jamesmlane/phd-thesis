
\section{A brief history of the Milky Way}

In this chapter is outlined the history of our understanding of the structure and properties of the Milky Way up to the first \textit{Gaia} data release. This body of knowledge also encompasses much of theory about the formation and evolution of the Milky Way, theory which becomes testable once \textit{Gaia} data can be used. I end the chapter by introducing the mathematical framework of distribution functions 

\subsection{Early history of Milky Way science}

% Fun early Milky Way shenanigans
The Milky Way, being observable with the unaided eye, has been known for the whole extent of human history. Interestingly, while the Milky Way features in the mythologies of most cultures it is rarely central and often of peripheral importance, especially when compared with resolved stellar associations and constellations, and particularly dynamic astronomical objects such as planets or comets. William Herschel was the first to perform a well-defined assay of the stellar contents of the Milky Way, the aspect ratio of the resulting distribution hinting at its true nature as a disk of stars \parencite{herschel1785}. The foresight of Immanuel Kant was also notable, who in 1755, predating Herschel, hypothesized that the Milky Way could be a flat disk of stars held together by mutual gravity, and embedded within it we would observe it as a band of light on the night sky. Another early astronomer whos intuition about the Milky Way was correct is al-Biruni, who in the 11th century poetically proposed that the Milky Way is ``a collection of countless fragments of the nature of nebulous stars".

% Integrating Milky Way and galaxy formation theory, ELS
Fast forward to the early 20th century, and our modern conceptualization of the Milky Way as a disk galaxy existing in a Universe full of other galaxies emerges. In the context of ongoing efforts to understand the stellar content of the Milky Way at the time, \textcite{roman50} demonstrated that stars of differing spectral line strength (indicating chemical abundances) have differing kinematics. This finding planted one of the many seeds that have blossomed into our modern understanding of the Milky Way as a collection of unique stellar populations. An important early work that integrated observations of the Milky Way into our understanding of galaxy formation is that of \textcite{eggen62}. They studied nearby dwarf stars, computing energies, angular momenta, and eccentricities, and found that a photometric metallicity proxy was inversely correlated with eccentricity. This indicated that more metal-poor stars are on more radial orbits, while metal-rich stars tend to be on nearly-circular orbits. They interpreted these results to mean that the Milky Way formed from a uniformly collapsing gas cloud, and that the oldest, most metal-poor stars in their sample formed out such infalling material.

% Searle & Zinn
\textcite{searle78} built upon this idea using measurements of the properties of Galactic globular clusters. They noted that globular cluster metallicity did not appear to be correlated with galactocentric distance, and that the clusters show variations in color-magnitude diagram morphology which is itself uncorrelated with metallicity. Since a natural prediction of the monolithic collapse model of \textcite{eggen62} would be a gradient in the metallicity of halo stars, the lack of such a gradient in the globular cluster data lead \textcite{searle78} to conclude that the globular clusters did not form in such a way. They instead hypothesize that if the Milky Way outer halo were assembled via the merging of proto-galactic fragments, that the resulting kinematics would be independent of metallicity, matching the globular cluster data.

% Pulling it together
Both of these seminal works had touched on key aspects of modern galaxy formation theory, as outlined in the previous section. A large reservoir of baryons condenses and collapses within a dark matter halo, as per \textcite{eggen62}. Additionally, the hierarchical nature of structure means that the merging and accretion of smaller proto-galactic fragments, and later dwarf galaxies, is a key growth-mode for disk galaxies like the Milky Way, as per \textcite{searle78}. While neither of these sets of authors had the complete picture, they were making largely correct, logical steps given the data they had.

\subsection{Milky Way anatomy}

% Introduce Bland-Hawthorn
The Milky Way is by most metrics a standard disk galaxy. It posseses the archetypal structural components found in other comparable disk galaxies: a multi-component stellar, gas, and dust disk; a centrally concentrated stellar bulge; an extended, approximately spherical stellar halo (although due to its faintness this is not readily observed in many other galaxies); and perturbations to these components such as spiral arms, and a central rotating bar. It has a stellar mass of about $6\times10^{10}$~\Msun, and resides in a dark matter halo with mass of the order $10^{12}$~\Msun.  A recent compilation of the structural and integrated properties of the Milky Way was presented by \textcite{bland-hawthorn16}, and drawing on that review here are summarized many of its key properties, particularly focusing on those pertaining to galactic archaeology.

% The stellar disk - overview and general kinematics
The eponymous part of any disk galaxy, the stellar disk is its heart and soul. The disk contains most of the stars in the galaxy, but it also hosts a prominant gas and dust component. The thinness of the stellar disk arises from the dynamics of the star-forming gas, which forms a rotating disk upon dissipational collapse to conserve angular momentum. The kinematics of the stars are inherited from this gas from which they form: the support is rotational, with the velocity dispersions typically much less than the circular velocity at any radius. Stars follow approximately epicyclic motion ontop of their comparatively fast circular orbits, making small radial excursions about their guiding center radii (the radius of a circular orbit with a given angular momentum) and above and below the disk. The rotation curve of the Milky Way is quite flat outside of the bulge region, having nominal values of $200-250$~km~s$^{-1}$ and being approximately $220-230$~km~s$^{-1}$ near the location of the Sun \parencite[e.g.][]{bovy12a,eilers19}.

% Gradients in the Milky Way disk
Beyond its kinematics the disk of the Milky Way is largely defined by a number of important gradients. There is an overall gradient in density \parencite[equivalently luminosity, as in other galaxies per e.g.][]{freeman70} which helps to give rise to the phenomenon of asymmetric drift, whereby a sample of stars in a small slice of radii appears to, on average, orbit slower than the local circular velocity. There are also important gradients in age, metallicity, as well as radial and vertical velocity dispersions \parencite[e.g.][]{bovy12d,bovy16b,mackereth19a}. Most of these gradients arise naturally in the context of inside-out galaxy formation, whereby the inner galaxy forms stars first and faster, leading to higher densities which drive higher dispersions, and also higher metallicity. Nonetheless they are important benchmarks when considering the formation history of the Galaxy.

% The thick disk
When considering the geometry of the disk, a particularly noteworthy observation emerges. The disk appears to be a superposition of a thin and thick component, with the thick component being noted by \textcite{gilmore83}. While the thick disk varies from the thin disk on the basis of its geometry (it also has a shorter scale length) as well as kinematics (obviously hotter) it is most unique in terms of its chemistry \textcite{gilmore95,bensby14}. The thick disk tends to be more metal poor than the thin disk, and it is enhanced in $\alpha$ elements, indicating a very old stellar population. There are many potential formation pathways for thick disks \parencite[e.g. see][]{robin14,minchev15}, which are typically divided into secular and external causes. Secular thick disk formation can be caused by the flaring of old, metal poor inner-disk stars during the natural redistribution of the angular momentum over the Galaxies lifetime \parencite{schoenrich09}, or they can also be naturally created earlier if the galaxy has a particularly hot, turbulant, early gas phase harrassed by mergers \parencite{brook04}. External factors that could cause or influence the creation of thick disk include: the preferential accretion of star-bearing satellites along the future disk plane \parencite{abadi03}, or heating of an extant disk by dwarf galaxy mergers \parencite{quinn93}. Since each of these scenarios is intimately linked with either old stellar populations in the Galaxy or with the accretion of dwarf satellites, constraining the genesis of the thick disk is of principle interest to near-field cosmologists and galactic archaeologists.

% Challenges with studying the thick disk
There are many challenges associated with the study of the thick disk. Principally, the variety of tracer populations and observational techniques (i.e. abundances, kinematics, spatial information) employed can yield sometimes complicated or contradictary answers \parencite{minchev15,kawata16}. For example, the thick disk clearly has two unique trends in age and metallicity \parencite{haywood13,hayden15}. But in terms of geometry the thin-thick disk dichotomy is not as clear, and it is better represented by a continuum of populations \parencite{bovy12e,bovy16b}. This is such that the thickest component of the thin disk is morphologically similar to the thinnest part of the thick disk \parencite{hayden17}. Hence it is better to refer to high- and low-$\alpha$ disk populations, with thin and thick disks referring to the geometric properties. Currently the most favoured scenario for the formation of the thick disk is via a well-attested major merger about 10~Gy ago (the nature of which will be identified and discussed extensively in the next section) which disrupted the gas and stellar disk of the early Milky Way \parencite{gallart19,belokurov20}, substantially heating the dynamics of the stars born during this epoch. While this is the prominant theory, it still faces challenges in explaining all observed properties of the thick disk and so the formation of the thick disk remains a somewhat open question.

% Disk perturbations: the bar
Another component of the disk, and important in the context of satellite galaxies and mergers, are perturbations such as the bar and spiral arms. The Milky Way bar is a prominant, $\sim~5$~kpc-long structure non-axisymmetric structure protruding from the galactic center at an angle of about $30~\degr$ towards the direction of Galactic rotation from the Galactic center-Sun line \parencite{wegg15}. The pattern speed of the bar has become well-constrained in the \textit{Gaia} era and is about 40~km~s$^{-1}$~kpc$^{-1}$ \parencite[e.g.][]{bovy19,sanders19}. While bar evolution is largely secular \parencite{athanassoula03}, they can form either secularly via disk instabilities \parencite{ostriker73}, or from tidal interactions and possibly mergers with nearby satellites \parencite{noguchi87,gerin90}. Therefore the bar acts as yet another stellar population in the Milky Way with ties to historical mergers and the broader cosmological context in which the Milky Way formed, and likely has connections to the aforementioned major merger that potentially formed the thick disk \parencite{fragkoudi20,merrow23}, to which we will return in the next section.

% Disk perturbations: spiral structure
It is also well-established that the Milky Way has numerous spiral arms, inferred most reliably across the disk from radio observations of molecular masers in star forming regions as well as HI emission \parencite{levine06,reid14,reid19}, and more recently near the Sun in stellar density \parencite{eilers19}. But both the secular evolution and formation of the Milky Way spiral structure is still highly uncertain. A major challenge with spiral structure is that it is likely transient \parencite[][reviews secular disk evolution, including the transient spiral structure]{sellwood14}, a behaviour for which evidence is seen in the Milky Way \parencite{hunt18,sellwood19}. The story of formation and evolution for spiral structure is similar in spirit as for the bar. The evolution is largely secular while the formation mechanisms are either secular, and particularly driven by the galactic bar and other inner-galaxy instabilities, or external driven by tidal interactions from passing satellites. Spiral structure therefore provides another probe of the Milky Way environment, albeit one for which we are less well-equiped to study at the present moment.

% Disk perturbations: the galactic phase-space spiral
A final noteworthy perturbation is the vertical phase space spiral discovered in disk star kinematics soon after the second \textit{Gaia} data release \parencite{antoja18}. The asymmetry of the spiral indicates a so-called ``bending mode'', or warping perturbation to the Milky Way disk. Many theories of its origin have been put forward and subsequently explored, including a recent passage by a satellite galaxy \parencite{laporte19}, the dynamical echo of the buckling instability of the galactic bar \parencite{khoperskov19}, or excitation via many small perturbations such as dark matter subhalos or transient dark matter wakes \parencite{tremaine23,grand23}. Exploring this perturbation across the extent of the galactic disk reveals its complexity, and it is now clear there is a mixture of symmetric and asymmetric phase spirals at different locations in the Milky Way \parencite{hunt22}, suggesting multiple perturbations at play. Further complicating the picture is that a recent satellite passage, the favoured scenario for the origin of the phase space spiral near the Sun, appears unable to convincingly reproduce observations \parencite{bennett21,bennett22}. Despite this, the undoubtable link between Milky Way disk disequilibrium and nearby satellites constitutes yet another important reason to consider the Milky Way disk in a broader cosmological context.

% The bulge
The centermost stellar component of the Milky Way, the bulge is a concentrated population of stars that encapsulates the inner few kpc of the Galaxy and extends above and below the disk \parencite{baade46}. Possessed by most spiral galaxies throughout the local Universe, bulges can be divided into two main categories: classical bulges are mostly spherical and have pressure-supported kinematics, while pseudobulges are elongated in the disk plane and have rotational kinematics. The Milky Way has been shown to have a pseudobulge, specifically a boxy or peanut-shaped (b/p) bulge \parencite{ness13a,ness13b,wegg13}, and a negligible classical bulge component \parencite{kunder16}. Moreover the b/p bulge appears to be an inner-Galaxy extension of the prominant mid-Galactic bar \parencite{wegg15}, bearing rotational kinematics as well as geometry and orbits consistent with formation via the bar-buckling instability \parencite{athanassoula05}. The chemistry of the bulge indicates a range of stellar populations, and specifically a widely observed enhancement in $\alpha$ elements indicating early rapid star formation \parencite{mcwilliam16,bensby17}. Interestingly, there are broad similarities between bulge and thick disk abundance patterns, which perhaps suggests a common origin for the two stellar populations, and in particular could support the idea that the b/p bulge formed from the buckling of the bar which itself is chemically similar to the inner thick disk \parencite{dimatteo14}.

% Potential formation scenarios for the bulge
The lack of a substantial classical bulge in the Milky Way is noteworthy since it is thought that these spherical, pressure-supported bulges are relics of early phases of aggressive, gas-rich star formation and accretion which disrupts the proto-disk \parencite{steinmetz95,samland03,obreja13}. But also complicating the picture is that a classically-formed stellar bulge may be secularly spun up by the Galactic bar, potentially masking its canonical kinematic traits and making it appear that lacks a classical stellar bulge when in fact its kinematics have been altered \parencite{saha12}. Parallel to the question of the existence of the classical bulge in the Milky Way is the study of the inner stellar halo, and particularly the degree to which it is an \textit{in-situ} component formed early in the Milky Way \parencite[e.g.][]{rix22}. These two putative stellar populations are linked by the similarities in their creation, both being old remnants of the earliest epochs of Milky Way formation. It may be the case that the classical stellar bulge is simply an inner extension of the \textit{in-situ} halo \parencite{perez-villegas17}. It is for these reasons that the study of the bulge continues to be an important pillar in the broader understanding of the history of the Milky Way, as well as the specific understanding of the stellar halo.


\subsection{The stellar halo of the Milky Way}

% Introduction to the stellar halo and why it is important
The final component of the Milky Way to consider is the stellar halo, which is deserving of its own section in light of its centrality to this thesis. Here will be presented an overview of the fundamentals of the stellar halo, focusing specifically on theory and accumulated knowledge prior to the launch of the \textit{Gaia} satellite. Much of the work presented here lays a superb foundation for more recent advancements enabled by \textit{Gaia}. The stellar halo is an approximately spheroidal assortment of stars embedded in and surrounding the Milky Way galaxy. Being completely devoid of star-forming gas, the stars in the stellar halo are characteristically old. They tend to be metal poor, consistent with their old ages, and typically move at high velocities with isotropic kinematics. The stellar halo gained promininance with the work of \textcite{eggen62}, who based their hypothesis of galaxy formation via dissipational collapse on the measured properties of fast-moving halo stars. \textcite{searle78} also based their arguments for galaxy assembly via hierarchical merging of smaller protogalactic fragments on the distribution and abundances of a prominant sub-population of the stellar halo: globular clusters. 

% Importance to galactic archaeology
The stellar halo is of principal importance to the endeavour of galactic archaeology for many reasons, and chief among them that it is the location in the Milky Way where dynamical timescales are longest. In other parts of the Galaxy stellar populations, such as the bulge or disk, mix together with one another, are subject to potentially chaotic perturbations from other components, or undergo complicated secular or externally driven evolution. While it is still possible to study these other stellar populations in the context of galaxy formation owing to their additional traits such as ages and chemical abundances, information can still be lost as stellar populations mix, and dynamical information about the formation of the Galaxy is undoubtedly erased. Additionally, any inferences are typically gleaned through the hazy lens of N-body or similar numerical models. This all contrasts with the stellar halo, where long dyamical timescales allow for stellar structures to maintain coherence over many billions of years. Stellar populations in the halo can still be confused and information lost via the same mechanisms as for inner-Galaxy populations, but in general at a much slower rate. The bottom line is that the stellar halo retains more valuable information from the earliest epochs of the Milky Way than other stellar populations, and so any effort to study the formation and evolution of the Milky Way should focus on the stellar halo.

% The dual stellar halo
Central to the study of the stellar halo is the notion that it is fundamentally of a ``dual'' nature \parencite{norris94,zolotov09}. It should contain two populations (in not necessarily equal amounts) of stars: one \textit{in-situ} population which was formed within the Galaxy, likely in its earliest epoch; and one \textit{accreted} population, formed in other smaller galaxies which have merged with and deposited their stellar contents in the halo of the Milky Way. Indeed, pre-\textit{Gaia} observations of the Milky Way stellar halo tended to reveal dualities in abundances, density, and kinematics \parencite[e.g.][]{kinman94,carollo07,nissen10,deason11}. As shall become evident when it comes to discussing more recent \textit{Gaia} findings, the accreted stellar halo is on solid footing while a picture of the \textit{in-situ} halo is only beginning to emerge and its true nature is somewhat controversial. Prevailing theories for the creation of the \textit{in-situ} halo include the remnants of the initial dissipational collapse and assembly of the proto-galaxy, the formation of stars from gas while it is being accreted (from the broader cosmic reservoir or stripped from an infalling satellite), and the disruption and heating of an early proto-disk \parencite{zolotov09,purcell10,font11,tissera13,cooper15}. One factor which complicates this simple dichotomy is that the very early Milky Way was likely built in large part via the merging of smaller proto-galactic fragments \parencite[i.e. within the standard galaxy formation frameworks of ][]{white78,white91}. So should these earliest stars be considered \textit{in-situ} or accreted? In practice recent works have distinguished these earliest stars as the ``proto-galaxy'' or ``young galaxy'' \parencite[e.g.][]{conroy22,belokurov22}. But nonetheless, the fact that it has become well-established that there are two separate formation mechanisms for distinct parts of the halo offers a valuable opportunity: study the accreted halo to learn about the impact of mergers, a primary growth-mode for spiral galaxies; and study the \textit{in-situ} halo to learn about the early conditions within the Milky Way itself.

% Differences in the properties of the dual halo
The accreted and \textit{in-situ} halo populations should differ in terms their abundances, kinematics, and densities, and within each population can exist trends based on the specific epoch and mechanism by which the stars were desposited into the halo. With regards to abundances, stars formed within the Milky Way benefited from its deep potential well, allowing the retention of much more enriched gas which leads to higher metallicities in formed stars, while accreted stars formed within the shallower potential wells of dwarf galaxies and globular clusters, which do not facillitate as rapid chemical enrichment for stars of a given age. This should lead to notable variations in [Fe/H], [$\alpha$/Fe], [Al/Fe], and [Mn/Fe] abundances \parencite{tumlinson10,zolotov10,hawkins15}. In terms of density profiles, the \textit{in-situ} halo is expected to be more centrally concentrated in the galaxy, and may overlap with the bulge region. But the specific location of these oldest stars formed within the Galaxy will reflect the early conditions of the Milky Way \parencite[e.g.][]{el-badry18}. Accreted stars, on the other hand, should be distributed in accordance with the geometry of the merger event that deposited them in the stellar halo, and are expected to occupy predominantly the outermost parts \parencite{abadi06}. Finally, the kinematics of the \textit{in-situ} population may be expected to again reflect the specific formation mechanism for this component. Accreted populations also inherit their kinematics from the merger event that deposited them in the stellar halo \parencite{bullock05,johnston08,cooper10}, and the retention of this information offers a powerful tool to reconstruct these mergers today. In aggregate, the accreted component should be roughly isotropic, as it is built of many mergers with assumed random accretion trajectories, and the kinematics should have a slight radial bias, reflecting the influence of dynamical friction which causes orbits of merging satellites to radialize.

% The accreted stellar halo as the potentially dominant component
Within the framework of $\Lambda$CDM and a Universe in which structure is hierarchical it is quite natural to assume that the stellar halo should be composed to a substantial degree of the remnants of mergers. Evidence from both cosmological and tailored N-body simulations suggests that the Milky Way should have experienced mergers with at least dozens, and up to 100-200 smaller satellites \parencite{abadi06,fakhouri10,font11,pillepich14}. Consistent with the halo mass function, a few of these satellites should be quite large (and therefore of higher metallicity) with stellar masses M$_{\star} \sim 10^{7}-10^{9}$~\Msun, and the majority should be smaller. Theoretical studies typically find that these few largest satellites dominate the resulting mass budget of the accreted stellar halo \parencite{bullock05,delucia08,cooper10}. Since dynamical friction acts more aggressively on more massive satellites, the debris from these few large accretion events should be concentrated in the inner halo. This prediction has been born out by findings from \textit{Gaia} that indeed show the inner stellar halo to composed of debris from a single major merger.

% Using kinematics to trace the individual mergers
The most interesting property of the merger remnants that constitute the accreted halo is that the kinematics of the debris from each individual event are approximately conserved (e.g. energy, angular momentum, actions) and reflect the circumstances of the merger \parencite{helmi00,johnston08}. This means that it is possible not only to glean the origin of this component of the stellar halo, but ideally to be able to compile a record of the individual mergers. This includes their masses, the time the merger occurred, and other relevant information about the constituent gas and stellar content of the accreted system. This is a very powerful prospect, since as explained in the previous section mergers have been linked to a myriad of formation and evolutionary mechanisms for many other components of the Milky Way, and by generalization other disk galaxies, and so it is worthwhile to characterize them. A good example of such an exercise in practice using pre-\textit{Gaia} data is the discovery of the ``Helmi streams'' by identifying substructure in orbital action space \parencite{helmi99}. The progenitor of the Helmi streams is thought to be a lower mass dwarf galaxy, perhaps akin to the extant Fornax dwarf spheroidal in the Milky Way stellar halo. As this exercise demonstrates, and as will be expounded in the next section, this catalogueing of mergers is not only possible but actively occurring in the \textit{Gaia} era.

% Sagittarius and other coherent accreted structures
So far the emphasis has been on ancient accretion events in the Milky Way. But testament to the recent and ongoing process of accretion can be seen now in observations of the stellar halo \parencite[e.g.][]{belokurov06}. A particularly significant example is the discovery and subsequent extensive study of the currently-merging Sagittarius dwarf galaxy \parencite[Sgr][]{ibata94}, the core of which is currently hidden behind the Galactic bulge. Sgr is on a polar orbit, and the tidal tails reflecting its doomed fate to be accreted into the Milky Way are evident above and below the disk \parencite{majewski03,belokurov06}. The merging of this dwarf is thought to contribute to the perturbation of the Galactic disk revealed by \textit{Gaia} \parencite[see the previous section][]{antoja18,laporte19}, and reinforces the important link between the evolution of the Milky Way and accretion events past and present.

% Other stellar halos 
Observations of the stellar halos of other nearby spiral galaxies, and particularly M31 (Andromeda), bolsters the notion that accretion is an ongoing process \parencite{mcconnachie09,martinez-delgado10}. The advantage of observing the stellar halos of other galaxies is that their surface brightness is much higher, as unlike the stellar halo of the Milky Way they are confined to a small solid angle and not spread over the whole sky. Such observations reveal the complexities of unmixed debris from recent accretion events: smaller disrupting systems produce arcs and tails during tidal stripping, while larger accretion events produce caustics, plumes, and shells \parencite[][provides a good overview of such phenomena]{johnston08}. The tidal tails emanating from stripped systems in the Milky Way are readily observable in the Milky Way due to their definiteness \parencite[e.g.][]{belokurov06}. But more diffuse caustics, plumes, shells, and other features are not easily inferred in the Milky Way and so observations of other galaxies coupled with N-body studies form the basis for our understanding of them. These are crucial to understand though, as old major mergers in the Milky Way will have undoubtedly left such complex signatures in the stellar halo, complicating the task of characterizing the properties and circumstances of the accretion event.

% Globular clusters
Up to this point the focus has been on the use of field stars to study the stellar halo, but a prominant sub-population of halo denizens are globular clusters \parencite[for catalogues see][]{harris96,baumgardt18}. Enigmatic objects, globular clusters are typically dense and bright (hence our extensive knowledge of them), but also very old and metal poor. Their ages are comparable to the age of the Universe, and they are thought to inhabit most galaxies. While their formation mechanisms are still not well understood \parencite{forbes18a}, they exhibit a number of useful internal properties such as having well-defined stellar populations, abundances, and kinematics \parencite[although many also have multiple populations with slight differences][]{milone22}. They also seem to obey broader population-level trends, such as having reasonably well-defined age-metallicity relations \parencite{forbes10,leaman13} corresponding to leaky-box chemical evolution models, or the fact that the number of clusters in a galaxy is correlated with its mass \parencite{harris13,forbes18b}. Finally, globular clusters belonging to dwarf spheroidals are largely expected to survive the process of their host merging with the Milky Way, owing to their dense nature \parencite{penarrubia09}. Indeed, examining the age-metallicity-kinematic trends of known globular clusters in the Milky Way reveals two populations: one with lower metallicities and extended halo-like kinematics, and one with higher metallicities and more concentrated disk-like kinematics. These two populations of GCs echo the \textit{in-situ} and accreted stellar halo, and indeed are expected to be of comparable nature.

% On the use of globular clusters
Despite a lack of satisfactory models that describe the formation of globular clusters, their empirical properties and the observed trends that they appear to hold to mean they are useful for galactic archaeology. Indeed, their ability to trace and describe accretion events has been demonstrated in the context of the merging Sgr dwarf spheroidal \parencite{law10}. In general, the kinematics of globular clusters should reflect the merger event, as for stars, and therefore accreted clusters should be roughly isotropically distributed with a range of kinematic properties \parencite[linking back to the hypothesis of ][]{searle78}. As with many of the other aspects of the stellar halo mentioned above, the \textit{Gaia} era substantially increases the amount of data with which they may be studied, and they will be used both to discover and characterize new and existing substructure in the stellar halo.

% Concluding paragraph
In this section we have seen how the stellar halo acts as a testing ground for many of the hypotheses of hierarchical galaxy formation in $\Lambda$CDM introduced in the first section of this introduction. Not only have we explored the theoretical links between the halo and galaxy formation theory, but directly touched on multiple aspects of the Milky Way today that can likely trace their origin, at least in part, to a merger or accretion event. These include the thick disk and the Galactic bar. Even before \textit{Gaia} data became available much progress had clearly been made, especially on the theoretical front. In the next section we will enter the \textit{Gaia} era and introduce many of the discoveries and advancements that it has facillitated.


\subsection{Mathematical tools for galactic dynamics}

To this point we have discussed numerous aspects of the history of the Milky Way Galaxy and its place in the cosmos. At this point we make must now delve into the mathematical framework which forms the basis for our understanding of the Galaxy. The standard reference for these materials is the textbook \textcite{binney08}, which will guide me here as well. I will introduce integrals of motion and actions as tools to efficiently describe orbits. I then move on to distribution functions, which are the primary model for the distribution of stellar kinematics in the Milky Way. These tools will be used throughout the thesis to describe the Milky Way, and to interpret the \textit{Gaia} and other data.

\subsubsection{Orbits and integrals of motion}

Among the many properties of a star, and of chief importance, is the nature of its orbit. The orbit of a star determines which part of the galaxy it exists in, and informs its past and present properties. In many ways galaxies can be thought of as being ``built'' out of orbits, a perspective which underpins many dynamical modelling approaches. Families of orbits intertwine to create the great stellar populations of the galaxy: the disk, bulge, and halo. While also defined by their unique chemistry and stellar ages, these populations are often most readily separated on the basis of the orbits of their constituent stars. It is with this in mind that I outline a framework for describing and labelling stellar orbits which informs their overall kinematics in much the same way their chemical composition may be informed by measured abundances. These descriptions will also end up providing the ingredients to construct distribution functions, which will be the subject of the next section.

The orbit of a star is dictated by the gravitational potential of the galaxy along with the position and velocity of the star at any given time. As these quantities are liable to vary throughout the orbit of a star, it is useful to instead turn to integrals of motion to describe orbits. These are quantities which are conserved along the orbit of a star, and so can be used as a unique set of labels for the orbit under consideration. The number of integrals of motion possessed by an orbit is determined by the nature of the potential. In the case of a spherically symmetric potential, four integrals of motion exist. First is the energy, given by 

\begin{equation}
    \label{ch1:eq:energy}
    E = \frac{1}{2}v^2 + \Phi(\mathbf{\mathrm{r}}),
\end{equation}

\noindent where $v$ is the magnitude of the velocity, and $\Phi(r)$ is the gravitational potential, only a function of the spherical radius $r$. The other three integrals of motion are the components of the angular momentum, given in vector form by

\begin{equation}
    \label{ch1:eq:angular-momentum}
    \mathbf{L} = \mathbf{r} \times \mathbf{v}\,,
\end{equation}

\noindent where $\mathbf{r}$ is the position vector and $\mathbf{v}$ is the velocity vector. Alternatively, the remaining three integrals of motion are often cast as the z-component of the angular momentum, the component perpendicular to this, and the magnitude of the angular momentum. In cylindrical coordinates the z-component of the angular momentum is given by

\begin{equation}
    \label{ch1:eq:z-angular-momentum}
    L_\mathrm{z} = R \, v_{T}\,,
\end{equation}

\noindent where $R$ is the cylindrical radius and $v_{T}$ the tangential velocity. The perpendicular component of the angular momentum is given by

\begin{equation}
    \label{ch1:eq:perpendicular-angular-momentum}
    L_{\perp} = \sqrt{ L^{2} - L_{\mathrm{z}}^{2} }\,,
\end{equation}

\noindent where $L$ is the total magnitude of the angular momentum. Regardless of the form, these four integrals of motion are sufficient to describe all of the unique orbits in a spherically symmetric potential.

But most galaxies, including the Milky Way and other spiral galaxies, do not have spherically symmetric potentials, and are instead better considered to be flattened or axisymmetric potentials. Such potentials permit three integrals of motion: the energy, the z-component of the angular momentum (assuming the potential is symmetric about the z-axis), and a quantity typically known as the ``third integral''. This third integral reflects the conservation of a unique combination of vertical and radial motions beyond the motions described simply by the conservation of energy and z-axis angular momentum (in other words two orbits may share $E$ and $L_{z}$ yet have different phase space trajectories). Among the many consequences of the conservation of third integral, a notable example can be seen when examining the trace of an orbit in the cylindrical $R-z$ plane, where the orbit will be confined to a characteristic area which is symmetric about the $z=0$ line \parencite[see figure 3.4 in][]{binney08}. The fact that the orbit is confined to this area, which is actually a toroidal volume in 3D space, is a reflection of the conservation of the third integral. One problem with the use of the third integral is that closed form expressions for it are limited to a few special potentials, such as the Henon potential which possesses a number of remarkable properties \parencite{henon59a,binney14e}, and so it is often not computed directly.

As potentials become more complicated the number of integrals of motion either decreases or they become more challenging to compute. Many such potentials are nontheless important to consider in the context of the Milky Way. Good examples are rotating potentials, such as those representing the Galactic bar, or triaxial potentials which are often used to model the dark matter halo. In some of these instances, such as for the dark halo potential, it is often sufficient to consider the potential to be of a simpler form, such as axisymmetric or spherical. For the Galactic bar advanced techniques are available that allow computation of conserved quantities in rotating frames of reference. More complicated time-varying behaviour or potentials that deviate significantly from axisymmetry generally require substantial simplifying assumptions to work with. A good example of such a potential that must be considered is the infalling Large Magellanic Cloud, which is currently merging with the Milky Way, and has been shown to have a significant impact on the dynamics of mid- and outer-halo stars and stellar structures \parencite[e.g.][]{erkal19}. In this instance perturbative techniques can be employed to handle the dynamical impact caused by the infalling satellite, but again I will defer discussion of these techniques. To summarize, as potentials deviate from time-indepent axisymmetry the accessibility of integrals of motion decreases and nearly always it is necessary to make simplifying assumptions to proceed. Throughout this thesis I generally will simplify the potential in order to access integrals of motion for analysis and to use as ingredients in distribution functions, and within each subsequent chapter will be relevant discussions of the implications of such choices where pertinant.

\subsubsection{Action-angle coordinates}

Within the framework of Hamiltonian mechanics, the equations of motion for an orbit are cast in terms of a canonical set of momenta and the corresponding conjugate coordinates. These canonical momenta are known as actions, $J_{i}$, and their behaviour is governed by Hamiltons equations, which also involve the Hamiltonian, $H$, and the aforementioned conjugate coordinates $\theta_{i}$. Hamiltons equations take the form

\begin{equation}
    \label{ch1:eq:hamiltons-equations}
\begin{split}
    \dot{J_{i}}= & -\frac{\partial H}{\theta{i}} \\
    \dot{\theta_{i}} = & \frac{\partial H}{J_{i}}\,.
\end{split}
\end{equation}

Assuming that the actions are integrals of motion, they are necessarily constant in time, and so the first of these equations is trivially satisfied, and the second yields simple linear time evolution for each of the angles $\theta_{i}$ 

\begin{equation}
    \label{ch1:eq:angle-evolution}
    \theta_{i}(t) = \theta_{i}(0) + \Omega_{i}t\,,
\end{equation}

\noindent involving a quantity $\Omega_{i} = \dot{\theta_{i}}$. Grounding ourselves in the reality that orbits are periodic we interpret these equations as those governing periodic motion. Therefore $\theta_{i}$ are akin to the phase of an orbit along its periodic trajectory, which may increase linearly without bound and yet only be relevant when considered modulo $2\pi$. Building on this we interpret the $\Omega_{i}$ as frequencies of this periodic motion. This equivalence is what gives rise to the standard names for these coordinates: action-angle variables, and their associated frequencies. It is standard to consider the three actions in cylindrical coordinates: $\{J_{R}, J_{\phi}, J_{z}\}$, and their associated angles $\{\theta_{R}, \theta_{\phi}, \theta_{z}\}$. It turns out that the actions are best defined using Poincar\'{e} invariants, which may be cast as line integrals along an orbit trajectory $\gamma$ 

\begin{equation}
    \label{ch1:eq:actions}
    J_{i} = \frac{1}{2\pi}\oint_{\gamma} p_{i}\, \mathrm{d}q_{i}\,,
\end{equation}

\noindent where $p_{i}$ and $q_{i}$ are components of the position and momentum in cylindrical coordinates. This form of the actions is useful for developing deeper intuition about their nature. As orbits explore a greater range of the configuration space $q_{i}$ (i.e. they orbit farther above and below the Galactic plane, or over a larger radial extent) at higher velocities $p_{i}$ the actions correspondingly increase. While equation~\eqref{ch1:eq:actions} is simple, the actual computation of actions using this equation or other methods can be challenging.

The actions only have closed form solutions for a small number of special potentials \parencite[again the potential of ][ for example]{henon59a} which in practice are only used as a part of toy models. Instead, the actions are nearly always computed using numerical techniques. In all cases the azimuthal action $J_{\phi} = L_\mathrm{z}$ is trivial, and only $J_{R}$ and $J_{z}$ require computation. Here I will outline a few of the most commonly used methods. The first is the spherical approximation, where we can directly draw on equation~\ref{ch1:eq:actions} to determine the actions. $J_{z} = L - L_{z}$ simply becomes the angular momentum net of the azimuthal component, and $J_{R}$ takes a slightly more complicated form involving an integral over the potential between the pericenter and apocenter. This spherical approximation is obviously suitable for spherical potentials, but also has applications as a starting point in axisymmetric potentials as well. Another approach is to employ the epicycle approximation and assume that the vertical motion above and below the disk may be decoupled from the radial motion towards and away from the Galactic center. Given this both a vertical and radial potential may be constructed and equation~\ref{ch1:eq:actions} used to compute the respective actions. This approach is reasonable for orbits which do not stray too far from the disk and which are not particularly eccentric, and is therefore well-suited for thin-disk orbits in the Milky Way.

The final approach - which is most favoured in modern galactic dynamics - is to model the underlying potential as a St\"{a}ckel potential, which implies that it is separable in confocal ellipsoidal coordinates $(u,v)$. This unusual choice of coordinate system is driven by observations of the cross-sections of orbits in the $R-z$ plane, which often appear to have boundaries approximately defined by lines of constant $u$ and $v$ \parencite[see figure 3.27 in ][]{binney08}. Therefore converting from cylindrical to prolate confocal coordinates gives a potential in which the $u$ and $v$ motions separate, and then an analog of equation~\eqref{ch1:eq:actions} may be used to compute the actions. This approach is particularly useful in the Milky Way as, in theory, it can reliably produce actions for a wide range of orbits, including those which venture far from the disk plane, and those on eccentric orbits. This is the approach I primarily use in this thesis and the specifics will be discussed in greater detail in the relevant chapters.

Actions and other integrals of motion introduced here represent orbits in the most fundamental manner, and are the core pieces of data used in Galactic astrophysics to study and model the various stellar populations of the Milky Way. Next, I will introduce distribution functions, the fundamental model for the distribution of phase space kinematics in realistic stellar populations, and which rely a great deal on these actions and integrals of motion.

\subsubsection{Distribution functions}

At a base level the DF is an expression of density in 6-dimensional phase space. When appropriately normalized, the integral of a DF $f$ over all space and all velocities is a constant related to the mass of the system

\begin{equation}
    \label{ch1:eq:df-normalization}
    M = \iint \mathrm{d}^3\mathbf{\mathrm{x}}\, \mathrm{d}^3\mathbf{\mathrm{v}}\, f( \mathbf{\mathrm{x}}, \mathbf{\mathrm{v}}) \,.
\end{equation}

\noindent Since DFs can describe both the positions and motions of stars in a probabilistic sense they are the natural tool for modelling discrete populations of stars in the Milky Way. The challenge is to construct DF models which are both physically motivated and which can be used to fit, study, and interpret the data.

As a starting point we consider a spherical stellar system. As is standard when studying DFs we work in terms of relative energies. The relative potential energy $\Psi$ is defined as 

\begin{equation}
    \label{ch1:eq:relative-potential-energy}
    \Psi = -\Phi + \Phi(\infty)\,,
\end{equation}

\noindent which is simply the negative of the standard gravitational potential offset such that the potential is zero at infinity. The relative energy $\mathcal{E}$ is then given by

\begin{equation}
    \label{ch1:eq:relative-energy}
    \mathcal{E} = \Psi - \frac{v^{2}}{2}\,,
\end{equation}

\noindent where $v$ is the magnitude of the velocity. Now, beginning with the simplest \textit{ansatz} that $f$ is a function only of $\mathcal{E}$, we can write the density of the system as an integral of the DF $f(\mathcal{E})$ over all velocities. If we work in spherical coordinates then the integral can be cast as 

\begin{equation}
    \label{ch1:eq:spherical-df-density}
    \nu(r) = 4\pi \int \mathrm{d} v \, v^{2} \, f(\mathcal{E}) = 4\pi \int \mathrm{d} \, \mathcal{E} \, f(\mathcal{E}) \, \sqrt{ 2(\Psi - \mathcal{E}) }\,.
\end{equation}

\noindent Then if we consider that we may cast the density as a function of $\Psi$ instead of $r$, recognize that the appropriate integration range of $\mathcal{E}$ is 0 to $\Psi$ (equivalent to integrating velocity from 0 to the escape velocity), and then differentiate with respect to $\Psi$ we arrive at the following expression

\begin{equation}
    \label{ch1:eq:spherical-df-density-derivative}
    \frac{1}{\sqrt{8}\pi} \frac{\mathrm{d} \nu}{\mathrm{d} \Psi} =  \int \mathrm{d} \mathcal{E} \, \frac{ f(\mathcal{E}) }{ \sqrt{\Psi - \mathcal{E}} }\,.
\end{equation}

\noindent This is an Abel integral equation which can be solved to yield a solution for the DF of the form

\begin{equation}
    \label{ch1:eq:eddington-inversion-df}
    f(\mathcal{E}) = \frac{1}{\sqrt{8}\pi^2} \left[ \int \frac{\mathrm{d} \Psi}{\sqrt{\Psi - \mathcal{E}}} \frac{\mathrm{d}^{2} \Psi}{\mathrm{d}\nu^{2}} \frac{1}{\sqrt{\mathcal{E}}} \left( \frac{\mathrm{d} \nu}{\mathrm{d} \Psi} \right)_{\Psi = 0} \right] \,.
\end{equation}

This inversion was first discovered by \textcite{eddington16}, and the DFs which it describes - those based solely on energy - are known as ergodic DFs. It is noteworthy to point out that the density $\nu$ and the potential $\Psi$ could be potential-density pairs, or they could be independent. $\nu$ expresses the density of the tracer, the constituent stars of the stellar population under consideration for example. $\Psi$ expresses the gravitational potential governing the dynamics of the tracer population, which could be sourced from $\nu$ for a self-gravitating system like a globular cluster or could be independent such the way in which the dark halo largely governs the dynamics of the stellar halo. The Eddington DF has a closed-form solution for a number of self-gravitating systems, including the widely used \textcite{hernquist90} potential among others. In general however, and especially when $\nu$ and $\Psi$ are not a potential-density pair, the Eddington inversion must be computed using numerical methods.

A DF, or a stellar population in general, may be described by its orbital anisotropy, $\beta$. The anisotropy of a stellar population is a measure of the degree to which the constituent orbits are preferentially radial, tangential, or neither. Its form, while not intuitive, has its roots in the spherical Jeans equation and is given by 

\begin{equation}
    \label{ch1:eq:anisotropy}
    \beta = 1 - \frac{\sigma_{\theta}^{2} + \sigma_{\phi}^{2}}{2\sigma_{r}^{2}}\,,
\end{equation}

\noindent where $\sigma_{[\theta, \phi, r]}$ are the velocity dispersions in spherical coordinates. The anisotropy has the domain $(-\infty,1]$ such that as $\beta$ approaches negative infinity the orbits become completely tangential, and as $\beta$ approaches 1 the orbits become completely radial. When $\beta=0$ the system is ergodic, and the orbits are isotropic or unbiased. Anisotropy is useful in the context of DFs since two stellar populations may share the same density profile and underlying potential, yet have different anisotropies reflecting different underlying kinematics. 

To construct DFs with variable anisotropy we use angular momentum, another integral of motion, as a second variable. Angular momentum is a natural choice for the purpose of varying the anisotropy since an orbit with a given energy has a maximum angular momentum when it is on a circular orbit, and its angular momentum decreases as it becomes more radial. The DF for a spherical system with constant anisotropy as a function of radius is

\begin{equation}
    \label{ch1:eq:constant-anisotropy-df}
    f(\mathcal{E}, L) = f_{1}(\mathcal{E}) L^{-2\beta}\,,
\end{equation}

\noindent where $f_{1}(\mathcal{E})$ is a function depending only on energy. The approach to solve for $f_{1}$ is similar in spirit to the approach used to create the Eddington inversion, albeit more complicated since the addition of angular momentum complicates the handling of the integral over velocity space. The end result is an integral equation of the form

\begin{equation}
    % \label{eq:AbelIntegral}
    \frac{ 2^{\beta-1/2} }{ 2\pi I_{\beta} } r^{2\beta}\nu = \int_{0}^{\Psi} \mathrm{d}\mathcal{E} \frac{ f_{1}(\mathcal{E}) }{ (\Psi-\mathcal{E})^{\beta-1/2} }\,.
\end{equation}
    
\noindent where $I_{\beta}$ is a constant given by

\begin{equation}
    I_{\beta} = \sqrt{\pi}\frac{\Gamma(1-\beta)}{\Gamma(3/2-\beta)}\,,
\end{equation}

\noindent with $\Gamma$ being the usual Gamma function. This is an Abel integral equation for $1/2 < \beta < 3/2$, giving a solution similar to that found in equation~\eqref{ch1:eq:eddington-inversion-df}. For $\beta < 1/2$ differentiating the integral with respect to $\Psi$ one or more times yields an Abel equation which may be similarly inverted. For certain potential-density pairs such as that pf \textcite{hernquist90}, and half-integer values of $\beta$, this equation has a closed-form solution. Otherwise the DF must be computed numerically, which is often computationally expensive. This presents a challenge when using these types of DFs, since any approach to fitting typically requires computing a likelihood or other objective function multiple times, and the DF would almost definitionally be included in such a function. Much of the work done in this thesis will involve the use of DFs with constant anisotropy and navigating these issues of computability will be a chief driver for certain choices made in the analysis, which will be expounded in subsequent chapters.

An anisotropic DF related to the Eddington DF was introduced independently by \textcite{osipkov79} and \textcite{merritt85}. These authors sought a DF with an anisotropy which increases with radius, which is aligned with galaxy formation theory. The Osipkov-Merritt DF employs a single pseudo-energy defined as 

\begin{equation}
    \label{ch1:eq:osipkov-merritt-pseudo-energy}
    \mathcal{Q} = \mathcal{E} - \frac{L^{2}}{2r_{a}^{2}}\,.
\end{equation}

\noindent where $r_{a}$ is a scale radius. It is possible to cast the integral over all velocities, which has been the starting point for the two previous DFs discussed, in terms of $\mathcal{Q}$, which actually greatly simplifies the resulting integral equation. The result is actually identical to the Eddington inversion, but with $\mathcal{Q}$ substituting for $\mathcal{E}$ and the density multiplied by a factor of $(1 + r^{2}/r_{a}^{2})$, and is therefore approximately equivalent to the Eddington DF in terms of computation strategy. The anisotropy of the Osipkov-Merritt DF varies between $\beta=0$ and $\beta=1$ according to the formula

\begin{equation}
    \label{ch1:eq:osipkov-merritt-anisotropy}
    \beta = \frac{r^{2}}{r^{2} + r_{a}^{2}}\,,
\end{equation}

and so clearly for $r=r_{a}$ the anisotropy is equal to $1/2$. The Osipkov-Merritt DF is a useful companion to the Eddington and constant anisotropy DFs, and one chapter of this thesis will be dedicated to comparing these DFs in a practical manner.

A final DF which has grown in popularity over the last decade is based on actions as opposed to energy and angular momentum. First formulated by \textcite{binney14d} and built upon by \textcite{posti15}, DFs of this type have no specifically defined form but a general structure would be a double power law function of the actions $\mathbf{J} = \{ J_{R}, J_{z}, J_{\phi} \}$ expressed as

\begin{equation}
    \label{ch1:eq:action-df}
    f(\mathbf{J}) = \frac{M}{(2\pi J_{0})^{3}} 
    \bigg[ 1 + \bigg( \frac{J_{0}}{h(\mathbf{J})} \bigg)^{\eta} \bigg]^{\Gamma/\eta} 
    \bigg[ 1 + \bigg( \frac{g(\mathbf{J})}{J_{0}} \bigg)^{\eta} \bigg]^{-B/\eta}
    \bigg[ 1 + \chi \tanh \frac{J_{\phi}}{J_{\phi,0}} \bigg]
    \,.
\end{equation}

\noindent Here, the first term normalizes the DF such that the mass is $M$. The second and third terms control the behaviour of the actions in the small and large regimes respectively (note that smaller and larger actions will roughly correspond to smaller and larger radii) with power law slopes $\Gamma$ and $B$, and a parameter controlling the steepness of the transition $\eta$. The functions $h(\mathbf{J})$ and $g(\mathbf{J})$ control the anisotropy and flattening of the model in the small and large action regimes respectively, and are given by 

\begin{equation}
\label{ch1:eq:action-df-flattening-anisotropy}
\begin{split}
    h(\mathbf{J}) = & g_{r} J_{r} + g_{z} J_{z} + (3-g_{r}-g_{z}) J_{\phi} \\
    g(\mathbf{J}) = & g_{r} J_{r} + g_{z} J_{z} + (3-g_{r}-g_{z}) J_{\phi}\,.
\end{split}
\end{equation}

\noindent The parameters $[g_{r}, g_{z}, h_{r}, h_{z}]$ control the anisotropy and flattening for each regime. The final term of the DF sets the overall sense of rotationa about the z-axis by weighting positive or negative values of $J_{\phi}$ with a hyperbolic tangent function, with the degree and scale of the rotation set by $\chi$ and $J_{\phi,0}$ respectively. Additional terms may be constructed and added to this DF which serve to exponentially truncate the model for large actions or to add a core for small actions, see \textcite{binney14d} for more details.

The usefulness of this DF is the fact that it is a direct function of three integrals of motion: the actions. This contrasts with the Eddington family of DFs, which are functions of one or two actions: energy and sometimes the angular momentum. Since axisymmetric potentials such as the Milky Way admit three integrals (energy, angular momentum, and the third integral), these action-based DFs can theoretically describe more realistic phase space distributions that the Eddington family cannot. Another benefit of action-based DFs is that they have a comparably simple form. But the downside is that, as previously discussed, actions are not typically easy to compute and often require approximations to be made, which contrasts with energy and angular momentum which are always well-defined.