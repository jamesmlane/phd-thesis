
\section{A brief history of the Milky Way}

In this chapter I outline the history of our understanding of the structure and properties of the Milky Way up to the first \textit{Gaia} data release. This body of knowledge also encompasses much of theory about the formation and evolution of the Milky Way, theory which becomes testable once \textit{Gaia} data can be used.

\subsection{Mathematical tools for describing the Milky Way}

To this point we have discussed numerous aspects of the history of the Milky Way galaxy and its place in the cosmos. At this point we make must now delve into the math which underpins our understanding of the galaxy. The standard reference for these materials is the textbook \cite{binney08}, which serves as our guide here as well. Here I first give an overview of galactic dynamics. I specifically introduce integrals of motion to describe orbits, which act both as useful labels as well as the ingredients with which distribution functions will be built.

\subsection{Orbits and integrals of motion}

Among the many properties of a star, and of chief importance, is the nature of its orbit. The orbit of a star determines which part of the galaxy it exists in, and informs greatly its past and present properties. In many ways galaxies can be thought of as built out of orbits. Families of orbits intertwine to create the great stellar populations of the galaxy: the disk, bulge, and halo. While also defined by their unique chemistry and stellar ages, these populations are most readily separated on the basis of the orbits of their constituent stars.

It is with this in mind that we outline a framework for describing the orbits of stars which will inform us of their orbits in much the same way their chemical composition is informed by measured abundances. These labels will also provide us with the ingredients to construct disstribution functions, which will be the subject of the next section.

\subsection{Distribution functions}

At the most fundamental level a DF is an expression of density in 6-dimensional phase space. When appropriately normalized, the integral over all space and all velocities is a constant, related to the mass of the system

\begin{equation}
    M = \iint \mathrm{d}^3\mathbf{\mathrm{x}}\, \mathrm{d}^3\mathbf{\mathrm{v}}\, f( \mathbf{\mathrm{x}}, \mathbf{\mathrm{v}} )
\end{equation}