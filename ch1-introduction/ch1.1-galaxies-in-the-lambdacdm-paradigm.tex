\section{Galaxies in the \texorpdfstring{$\Lambda$CDM}{LambdaCDM} paradigm}

I begin with a look at the broadest cosmological scales, to attempt to contextualize the fundamental unit of the Universe: the galaxy, within the greater whole of which it is a part. This involves covering the $\Lambda$CDM paradigm, which is the current standard model of cosmology including the Big Bang theory, and highlighting how many of its central features are integral to the study of galaxies. I pay specific attention to dark matter, emphasizing its core role in the formation and resulting structure of galaxies. Finally, I discuss the modern tenets of galaxy formation theory, from the collapse of overdensities in the early Universe, to the assembly of structures resembling galaxies, and their resulting evolution over cosmic time.

\subsection{\texorpdfstring{$\Lambda$CDM}{LambdaCDM}: a tapestry for galaxy physics}

% Introduction to Lambda-CDM
The fruit of nearly a century of scientific investigation into the nature and behaviour of the Universe on its largest scales: $\Lambda$CDM is the concordance model of cosmology, widely accepted today as most accurate accounting of the evolution of the Universe from shortly after the Big Bang. At its core, $\Lambda$CDM describes a spatially flat, expanding Universe in which the energy density at the present day is dominated by three components: dark energy, dark matter, and baryonic matter. Our understanding of $\Lambda$CDM is buoyed by studies spanning nearly all wavelengths of light and epochs cosmic history. In the local Universe an early landmark study was the work of \textcite{hubble29} who firmly established the metric expansion of space. Descending from this work are the studies of supernovae Type IA nearly 60 years later which discovered the accelerating expansion of space due to dark energy \parencite{riess98,perlmutter99}. Studies of the cosmic microwave background (CMB) provides the early-Universe anchor for our understanding of $\Lambda$CDM, and the science produced by observatories such as WMAP and Planck have helped to define the current canonical set of cosmological parameters that define $\Lambda$CDM.

% Dark matter
Dark matter plays a key role in the $\Lambda$CDM model, comprising roughly 5/6 of the gravitating energy density of the Universe. Early conceptualizations of dark matter can be traced to the work of \textcite{kapteyn1922}, \textcite{oort32}, \textcite{zwicky33}, and \textcite{babcock39}, who all found that the kinematics of stars and galaxies could not seemingly be accounted for by the gravitational budget of the associated luminous matter. But it was the measurements of the rotation curve of M31 by \textcite{rubin70}, later extended to multiple galaxies by \textcite{rubin80}, which cemented the ``missing mass'' problem in astrophysics. Dark matter: a gravitating, non-luminous, baryonic or non-baryonic matter was the early preferred resolution to this problem. Alternatively, theories that propose a modification to gravity in the weak regime, originally hypothesized by \textcite{milgrom83}, have been popular due to their ability to resolve the missing mass problem and tackle certain other issues in theoretical physics. But these have lost traction in recent decades due to accumulation of such a wide range of evidence for dark matter that cannot be reconciled with any one theory of modified gravity. The work here in my thesis assumes the validity of dark matter, and I base much of my research on previous studies and results which do the same.

% Theoretical considerations for the dark matter halo
In parallel to observational validations of the missing mass problem, influential theoretical works by \textcite{ostriker73} assessing the stability of galaxy disks, and \textcite{einasto74} examining the dynamics of cluster galaxies both concluded that massive, approximately spherical dark halos in which galaxies and clusters exist could populate the Universe. Within a few years the concept of the dark matter halo had been integrated into theories of galaxy formation via clustering and dissipation \parencite{white78}. Theoretical work continued over the next few decades, driven in large part by the emerging field of N-body simulations, which further developed galaxy formation theory in the context of dark matter \parencite{white91}. Today dark matter remains an integral part of $\Lambda$CDM and is central to our understanding of cosmological structure as well as galaxy physics \parencite[e.g.][]{frenk12}.

% Dark matter is cold
One of the principle ways in which dark matter candidates are distinguished is their thermal velocities at early times, which governs the length scales at which dark matter clusters. Denoting them with an analogous ``temperature'' lead to the proposition of cold, warm, and hot dark matter. Hot dark matter (e.g. light neutrinos) became disfavoured when a combination of large scale galaxy redshift surveys and N-body studies demonstrated that the observed distribution of galaxies is inconsistant with a Universe dominated by hot dark matter in which galaxies would largely be expected to form only in clusters \parencite{white83}. Cold dark matter (CDM, e.g. axions or weakly interacting massive particles) produces results in excellent agreement with the apparent large scale distribution of galaxies, and is today the favoured set of dark matter candidates, hence the CDM in $\Lambda$CDM. Warm dark matter does remain plausible, and observational tests to distinguish between cold and warm dark matter occur on the scale of small dwarf galaxies and within the halos of galaxies like the Milky Way, two frontiers which are actively explored today.

% Conclude this subsection
Together, these observations and theoretical efforts have built a reasonably self-consistent picture of the cosmos, but $\Lambda$CDM and the accompanying Big Bang Theory are not without challenges. There are still open questions which may be solved by physics that supplements these models, such as inflation: the theory that space experienced exponential expansion in the very early Universe \parencite{guth81,linde82}. There is also currently a disagreement in the scientific community about the value of the Hubble parameter \parencite[see][]{divalentino21}, which appears to take a different value based on early or late Universe measurments, and may hint at new physics. But nonetheless $\Lambda$CDM provides a sublime explanatory framework within which we can delve deep into the rich veins of galactic astrophysics, and has justifiably earned its place in the pantheon of great physical theories.

\subsection{Hierarchical structure and the standard theory of galaxy formation}

% Introduction and primordial overdensities
Structure forms in the $\Lambda$CDM model from overdensities in the early Universe which are seeded by minute quantum fluctuations. One of the attractive features of inflation theory is that it provides a quantum field to generate such fluctuations \parencite{hawking82,guth82,starobinsky82}. These density perturbations are nearly scale invariant (small looks similar to large), adiabatic (radiation and matter perturbations are in thermal equilibrium), and of a small (but tunable) amplitude, with a power spectrum approximating a power law in wavenumber. The evolution of these perturbations is complex and depends on a variety of factors at different times, but much of the underlying physics and assumptions is backed up by CMB observations, which probe these fluctuations in the era of recombination, 380,000 years after the Big Bang and subsequent inflationary epoch.

% Collapse of overdensities and the halo mass function
The overdensities begin to gravitationally collapse when their typical density exceeds a critical threshold when compared to the background, with smaller structures forming first because they decouple from the Hubble flow earlier, more easily surpass the critical density, and collapse to a virialized state faster. The dark matter, being collisionless, leads the gravitational collapse and forms halos: the seeds and homes of future galaxies. It is at this point that the distinction between cold, warm, and hot dark matter places a limit on the smallest halos which may collapse, since those smaller than the typical distance dark matter particles travels up to this point, dictated by their thermal velocity, cannot form. The distribution of halos by mass has a power law shape, and is therefore \textit{hierarchical}: there are many more smaller halos than larger ones \parencite{gao04,tinker08}. This halo mass function was famously first derived via a simple model by \textcite{press74}, producing a form that has been built upon in intervening years \parencite{bond91,sheth99,sheth01} yet shows generally good agreement with observational and N-body assessments of dark matter halo statistics.

% The structure of dark matter halos
The behaviour of the collapsing dark matter overdensities and the hierarchical nature of the halo mass function gives rise to a unique structure for the halos \parencite[e.g.][]{frenk85,frenk88}. Since the central regions of a dark matter halo will collapse and virialize first, halos are said to form ``inside out''. Additionally, since a large number of small halos collapse before large ones, large halos will end up being built in part out of the merging and accretion of smaller structures in a process known as ``bottom-up'' formation (as opposed to a top-down formation in which large halos form first and fragment). These are key underlying precepts for galaxy formation theory that are associated with some of the principle growth modes of galaxies that we will come to momentarily. In addition to the studies in the latter part of the 20th century that sought to describe the halo mass function was also work to describe the internal structure of halos themselves, primarily using N-body simulations \parencite[e.g.][]{efstathiou88,dubinski91}. The use of N-body simulations as opposed to any semi-analytic technique was crucial, as it naturally allows for hierarchical merging and other complex behaviour integral to CDM. A crucial set of insights were presented by \textcite{navarro96} and \textcite{navarro97}, who found that dark matter halos have a ``universal'' inner structure, which is independent of the underlying cosmology and initial spectrum of density fluctuations. These NFW halos have self-similar circular velocity profiles and densities profiles $\rho(r)$, which take the form

\begin{equation}
    \label{ch1:eq:nfw}
    \rho(r) = \frac{\rho_{0}}{(r/r_{s})(1 + r/r_{s})^{2}}
\end{equation}

\noindent where $\rho_{0}$ and $r_{s}$ are a scale density and radius respectively \parencite[note this is a special case of the density profile of ][with {$[\alpha,\beta]=[1,3]$}]{dehnen93}. This profile remains in wide use today and has been shown to be a good fit for dark halos even in modern N-body simulations which include baryons, although alternative models exist \parencite[e.g.][]{einasto65}. It is the default fitting form for dark halos in observations as well, and most of the standard Milky Way potential models use the NFW model \parencite[e.g.][]{bovy15,mcmillan17}. Despite the success of the model in simulations, observations suggest that there are a variety of dark matter halo density profile shapes, particularly in the inner regions of dwarf galaxies which are often cored in comparison to the cuspy expectation from simulations. This ``cusp-core'' problem \parencite[see][for a summary]{bullock17} is one among many small-scale challenges facing $Lambda$CDM, but great steps have been taken in recent years to reconcile these issues in support of the concordance model, typically by invoking the complex physics of baryons and resulting interplay with dark matter.

% The standard picture of galaxy formation
Within the spectrum of dark matter halos described above occurs galaxy formation and then evolution over the subsequent long extent of cosmic history. The standard picture of galaxy formation from hierarchical structure in $\Lambda$CDM was first outlined by \textcite{white78} and extended to a semi-analytic form by \textcite{white91}. At a high level the process occurs over two stages: first dark matter halos form according to the processes outlined above, and then gas accumulates in the potential well and collapses dissipatively, forming the baryonic component of the galaxy. The detailed accounting of galaxy formation obviously includes a myriad of other important effects, including stellar and active galactic nuclei feedback, ongoing accretion of gas from a broader cosmological reservoir, and the accretion and merging of extant galaxies.

% Spiral and elliptical galaxies
How well does this broad picture account for the observed population of galaxies? It has been long known that there are two main types of galaxies, most easily defined by their morphologies: disks and ellipticals \parencite[e.g. the famous ``tuning fork'' classification of][]{hubble26}. The two-stage model of galaxy formation in hierarchical CDM universes naturally generates these two principle populations of large galaxies \parencite{fall79,efstathiou83,blumenthal84}. Most galaxies, and particularly those in more isolated environments, will become disks. As the gas cools and condenses in the gravitational well of the dark matter halo it conserves any initial angular momentum, and may be torqued by other nearby halos or early mergers, thus creating a rotationally supported, star-forming disk. Ellipticals will generally form out of the merging of two or more disk galaxies of comparable size, which consumes any remaining gas and randomizes the orbits of constituent stars, and will therefore be more prevalent in clusters \parencite{toomre72,gerhard81,barnes88}.

% Growth modes for disk galaxies
Focusing more specifically on the formation of disk galaxies, there are three principle modes by which disk galaxies grow. The first mode is that mentioned above, the condensation of baryons from an initial reservoir which grows from the same overdensities that seeded the host CDM halo. Much is still not known about this process, and while it may be studied with N-body simulations, obtaining observational data for constraints is one of the current goals of the \textit{James Webb Space Telescope} as well as large stellar surveys within the Milky Way. It is clear that feedback among the baryons, the transfer of energy and angular momentum between baryons and dark matter, and the still-cloudy physics of the first stars are all important for setting many of the characteristic features of the structure of galactic disks \parencite{vandenbosch01}. Notable features of disk galaxies which are likely sourced from these physics of formation are their metallicity gradients, their approximately exponential vertical and radial structure, as well as the characteristic bulge-disk morphology.

% Ongoing accretion of gas, and merging
At later times baryons in the form of diffuse gas continue to be accreted from a broader cosmological reservoir, and are also recycled via the diffuse, hot halo mediated by feedback within the galaxy. Correct treatment of these processes requires detailed knowledge of both galactic feedback physics as well as the physics of the diffuse, hot, gaseous halo, both of which continue to be developed and improved among modern N-body suites using new observations \parencite[e.g.][]{crain23}. The other key mode by which galaxies grow at later times is by merging with and accreting smaller structures such as dwarf galaxies \parencite{lacey93}. Here, galaxies will ingest not only any gas remaining in the accreted system, but stars which have already formed as well as dark matter.

% Galactic archaeology
One of the main goals of galactic astrophysics within the Milky Way is not only to generate constraints to aid in the formulation of a broader theory of galaxy formation, but to specifically trace and understand the evolution of the Milky Way. A popular modern framework for this endeavour is known as ``galactic archaeology'', popularized by \textcite{freeman02}, it posits that we can use the present-day properties of long-lived Milky Way stars to infer its formation and evolutionary history. An intoxicating end-goal, still beyond our reach and potentially forever infeasible, is to be able to link each star in the Milky Way to an element of some proto-galactic fragment that was involved in the initial formation of our galaxy, or was otherwise accreted. 

% Fossils in the Milky Way and its stellar halo
While a detailed accounting of each element of the initial cosmic overdensity that gave rise to the Milky Way may be fanciful, the concept of ``fossils'': remnants of past events and physical processes that shaped the Galaxy which remain today in the form of groups of stars or trends among stars, is central to galactic archaeology. A standard type of fossil of great interest today, which traces one of the principle galactic growth modes outlined above, are the stellar debris of past and ongoing accretion events whereby a dwarf galaxy merges with the Milky Way. The remainder of this introduction will largely be focused on setting the stage for the study of these accretion remnants, including the context of our Milky Way galaxy in which they are studied, the data used to observe them, and the models used to constrain their properties.