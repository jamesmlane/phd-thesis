
\section{A brief history of the Milky Way}

In this section is outlined the history of our understanding of the structure and properties of the Milky Way up to the first \textit{Gaia} data release. This body of knowledge also encompasses much of theory about the formation and evolution of the Milky Way, theory which becomes testable once \textit{Gaia} data can be used. I end the chapter by introducing the mathematical framework of distribution functions 

\subsection{Early history of Milky Way science}

% Fun early Milky Way shenanigans
The Milky Way, being observable with the unaided eye, has been known for the whole extent of human history. Interestingly, while the Milky Way features in the mythologies of most cultures it is rarely central and often of peripheral importance, especially when compared with resolved stellar associations and constellations, and particularly dynamic astronomical objects such as planets or comets. William Herschel was the first to perform a well-defined assay of the stellar contents of the Milky Way, the aspect ratio of the resulting distribution hinting at its true nature as a disk of stars \parencite{herschel1785}. The foresight of Immanuel Kant was also notable, who in 1755, predating Herschel, hypothesized that the Milky Way could be a flat disk of stars held together by mutual gravity, and embedded within it we would observe it as a band of light on the night sky. Another early astronomer whos intuition about the Milky Way was correct is al-Biruni, who in the 11th century poetically proposed that the Milky Way is ``a collection of countless fragments of the nature of nebulous stars".

% Integrating Milky Way and galaxy formation theory, ELS
Fast forward to the early 20th century, and our modern conceptualization of the Milky Way as a disk galaxy existing in a Universe full of other galaxies emerges. In the context of ongoing efforts to understand the stellar content of the Milky Way at the time, \textcite{roman50} demonstrated that stars of differing spectral line strength (indicating chemical abundances) have differing kinematics. This finding planted one of the many seeds that have blossomed into our modern understanding of the Milky Way as a collection of unique stellar populations. An important early work that integrated observations of the Milky Way into our understanding of galaxy formation is that of \textcite{eggen62}. They studied nearby dwarf stars, computing energies, angular momenta, and eccentricities, and found that a photometric metallicity proxy was inversely correlated with eccentricity. This indicated that more metal-poor stars are on more radial orbits, while metal-rich stars tend to be on nearly-circular orbits. They interpreted these results to mean that the Milky Way formed from a uniformly collapsing gas cloud, and that the oldest, most metal-poor stars in their sample formed out such infalling material.

% Searle & Zinn
\textcite{searle78} built upon this idea using measurements of the properties of Galactic globular clusters. They noted that globular cluster metallicity did not appear to be correlated with galactocentric distance, and that the clusters show variations in color-magnitude diagram morphology which is itself uncorrelated with metallicity. Since a natural prediction of the monolithic collapse model of \textcite{eggen62} would be a gradient in the metallicity of halo stars, the lack of such a gradient in the globular cluster data lead \textcite{searle78} to conclude that the globular clusters did not form in such a way. They instead hypothesize that if the Milky Way outer halo were assembled via the merging of proto-galactic fragments, that the resulting kinematics would be independent of metallicity, matching the globular cluster data.

% Pulling it together
Both of these seminal works had touched on key aspects of modern galaxy formation theory, as outlined in the previous section. A large reservoir of baryons condenses and collapses within a dark matter halo, as per \textcite{eggen62}. Additionally, the hierarchical nature of structure means that the merging and accretion of smaller proto-galactic fragments, and later dwarf galaxies, is a key growth-mode for disk galaxies like the Milky Way, as per \textcite{searle78}. While neither of these sets of authors had the complete picture, they were making largely correct, logical steps given the data they had.

\subsection{Milky Way anatomy}

% Introduce Bland-Hawthorn
The Milky Way is by most metrics a standard disk galaxy. It posseses the archetypal structural components found in other comparable disk galaxies: a multi-component stellar, gas, and dust disk; a centrally concentrated stellar bulge; an extended, approximately spherical stellar halo (although due to its faintness this is not readily observed in many other galaxies); and perturbations to these components such as spiral arms, and a central rotating bar. It has a stellar mass of about $6\times10^{10}$~\Msun, and resides in a dark matter halo with mass of the order $10^{12}$~\Msun.  A recent compilation of the structural and integrated properties of the Milky Way was presented by \textcite{bland-hawthorn16}, and drawing on that review here are summarized many of its key properties, particularly focusing on those pertaining to galactic archaeology \parencite[see also][]{ivezic12}.

% The stellar disk - overview and general kinematics
The eponymous part of any disk galaxy, the stellar disk is its heart and soul. The disk contains most of the stars in the galaxy, but it also hosts a prominant gas and dust component. The thinness of the stellar disk arises from the dynamics of the star-forming gas, which forms a rotating disk upon dissipational collapse to conserve angular momentum. The kinematics of the stars are inherited from this gas from which they form: the support is rotational, with the velocity dispersions typically much less than the circular velocity at any radius. Stars follow approximately epicyclic motion ontop of their comparatively fast circular orbits, making small radial excursions about their guiding center radii (the radius of a circular orbit with a given angular momentum) and above and below the disk. The rotation curve of the Milky Way is quite flat outside of the bulge region, having nominal values of $200-250$~km~s$^{-1}$ and being approximately $220-230$~km~s$^{-1}$ near the location of the Sun \parencite[e.g.][]{bovy12a,eilers19}.

% Gradients in the Milky Way disk
Beyond its kinematics the disk of the Milky Way is largely defined by a number of important gradients. There is an overall gradient in density \parencite[equivalently luminosity, as in other galaxies per e.g.][]{freeman70} which helps to give rise to the phenomenon of asymmetric drift, whereby a sample of stars in a small slice of radii appears to, on average, orbit slower than the local circular velocity. There are also important gradients in age, metallicity, as well as radial and vertical velocity dispersions \parencite[e.g.][]{bovy12d,bovy16b,mackereth19a}. Most of these gradients arise naturally in the context of inside-out galaxy formation, whereby the inner galaxy forms stars first and faster, leading to higher densities which drive higher dispersions, and also higher metallicity. Nonetheless they are important benchmarks when considering the formation history of the Galaxy.

% The thick disk
When considering the geometry of the disk, a particularly noteworthy observation emerges. The disk appears to be a superposition of a thin and thick component, with the thick component being noted by \textcite{gilmore83}. While the thick disk varies from the thin disk on the basis of its geometry (it also has a shorter scale length) as well as kinematics (obviously hotter) it is most unique in terms of its chemistry \textcite{gilmore95,bensby14}. The thick disk tends to be more metal poor than the thin disk, and it is enhanced in $\alpha$ elements, indicating a very old stellar population. There are many potential formation pathways for thick disks \parencite[e.g. see][]{robin14,minchev15}, which are typically divided into secular and external causes. Secular thick disk formation can be caused by the flaring of old, metal poor inner-disk stars during the natural redistribution of the angular momentum over the Galaxies lifetime \parencite{schoenrich09}, or they can also be naturally created earlier if the galaxy has a particularly hot, turbulant, early gas phase harrassed by mergers \parencite{brook04}. External factors that could cause or influence the creation of thick disk include: the preferential accretion of star-bearing satellites along the future disk plane \parencite{abadi03}, or heating of an extant disk by dwarf galaxy mergers \parencite{quinn93}. Since each of these scenarios is intimately linked with either old stellar populations in the Galaxy or with the accretion of dwarf satellites, constraining the genesis of the thick disk is of principle interest to near-field cosmologists and galactic archaeologists.

% Challenges with studying the thick disk
There are many challenges associated with the study of the thick disk. Principally, the variety of tracer populations and observational techniques (i.e. abundances, kinematics, spatial information) employed can yield sometimes complicated or contradictary answers \parencite{minchev15,kawata16}. For example, the thick disk clearly has two unique trends in age and metallicity \parencite{haywood13,hayden15}. But in terms of geometry the thin-thick disk dichotomy is not as clear, and it is better represented by a continuum of populations \parencite{bovy12e,bovy16b}. This is such that the thickest component of the thin disk is morphologically similar to the thinnest part of the thick disk \parencite{hayden17}. Hence it is better to refer to high- and low-$\alpha$ disk populations, with thin and thick disks referring to the geometric properties. Currently the most favoured scenario for the formation of the thick disk is via a well-attested major merger about 10~Gy ago (the nature of which will be identified and discussed extensively in the next section) which disrupted the gas and stellar disk of the early Milky Way \parencite{gallart19,belokurov20}, substantially heating the dynamics of the stars born during this epoch. While this is the prominant theory, it still faces challenges in explaining all observed properties of the thick disk and so the formation of the thick disk remains a somewhat open question.

% Disk perturbations: the bar
Another component of the disk, and important in the context of satellite galaxies and mergers, are perturbations such as the bar and spiral arms. The Milky Way bar is a prominant, $\sim~5$~kpc-long structure non-axisymmetric structure protruding from the galactic center at an angle of about $30~\degr$ towards the direction of Galactic rotation from the Galactic center-Sun line \parencite{wegg15}. The pattern speed of the bar has become well-constrained in the \textit{Gaia} era and is about 40~km~s$^{-1}$~kpc$^{-1}$ \parencite[e.g.][]{bovy19,sanders19}. While bar evolution is largely secular \parencite{athanassoula03}, they can form either secularly via disk instabilities \parencite{ostriker73}, or from tidal interactions and possibly mergers with nearby satellites \parencite{noguchi87,gerin90}. Therefore the bar acts as yet another stellar population in the Milky Way with ties to historical mergers and the broader cosmological context in which the Milky Way formed, and likely has connections to the aforementioned major merger that potentially formed the thick disk \parencite{fragkoudi20,merrow23}, to which we will return in the next section.

% Disk perturbations: spiral structure
It is also well-established that the Milky Way has numerous spiral arms, inferred most reliably across the disk from radio observations of molecular masers in star forming regions as well as HI emission \parencite{levine06,reid14,reid19}, and more recently near the Sun in stellar density \parencite{eilers19}. But both the secular evolution and formation of the Milky Way spiral structure is still highly uncertain. A major challenge with spiral structure is that it is likely transient \parencite[][reviews secular disk evolution, including the transient spiral structure]{sellwood14}, a behaviour for which evidence is seen in the Milky Way \parencite{hunt18,sellwood19}. The story of formation and evolution for spiral structure is similar in spirit as for the bar. The evolution is largely secular while the formation mechanisms are either secular, and particularly driven by the galactic bar and other inner-galaxy instabilities, or external driven by tidal interactions from passing satellites. Spiral structure therefore provides another probe of the Milky Way environment, albeit one for which we are less well-equiped to study at the present moment.

% Disk perturbations: the galactic phase-space spiral
A final noteworthy perturbation is the vertical phase space spiral discovered in disk star kinematics soon after the second \textit{Gaia} data release \parencite{antoja18}. The asymmetry of the spiral indicates a so-called ``bending mode'', or warping perturbation to the Milky Way disk. Many theories of its origin have been put forward and subsequently explored, including a recent passage by a satellite galaxy \parencite{laporte19}, the dynamical echo of the buckling instability of the galactic bar \parencite{khoperskov19}, or excitation via many small perturbations such as dark matter subhalos or transient dark matter wakes \parencite{tremaine23,grand23}. Exploring this perturbation across the extent of the galactic disk reveals its complexity, and it is now clear there is a mixture of symmetric and asymmetric phase spirals at different locations in the Milky Way \parencite{hunt22}, suggesting multiple perturbations at play. Further complicating the picture is that a recent satellite passage, the favoured scenario for the origin of the phase space spiral near the Sun, appears unable to convincingly reproduce observations \parencite{bennett21,bennett22}. Despite this, the undoubtable link between Milky Way disk disequilibrium and nearby satellites constitutes yet another important reason to consider the Milky Way disk in a broader cosmological context.

% The bulge
The centermost stellar component of the Milky Way, the bulge is a concentrated population of stars that encapsulates the inner few kpc of the Galaxy and extends above and below the disk \parencite{baade46}. Possessed by most spiral galaxies throughout the local Universe, bulges can be divided into two main categories: classical bulges are mostly spherical and have pressure-supported kinematics, while pseudobulges are elongated in the disk plane and have rotational kinematics. The Milky Way has been shown to have a pseudobulge, specifically a boxy or peanut-shaped (b/p) bulge \parencite{ness13a,ness13b,wegg13}, and a negligible classical bulge component \parencite{kunder16}. Moreover the b/p bulge appears to be an inner-Galaxy extension of the prominant mid-Galactic bar \parencite{wegg15}, bearing rotational kinematics as well as geometry and orbits consistent with formation via the bar-buckling instability \parencite{athanassoula05}. The chemistry of the bulge indicates a range of stellar populations, and specifically a widely observed enhancement in $\alpha$ elements indicating early rapid star formation \parencite{mcwilliam16,bensby17}. Interestingly, there are broad similarities between bulge and thick disk abundance patterns, which perhaps suggests a common origin for the two stellar populations, and in particular could support the idea that the b/p bulge formed from the buckling of the bar which itself is chemically similar to the inner thick disk \parencite{dimatteo14}.

% Potential formation scenarios for the bulge
The lack of a substantial classical bulge in the Milky Way is noteworthy since it is thought that these spherical, pressure-supported bulges are relics of early phases of aggressive, gas-rich star formation and accretion which disrupts the proto-disk \parencite{steinmetz95,samland03,obreja13}. But also complicating the picture is that a classically-formed stellar bulge may be secularly spun up by the Galactic bar, potentially masking its canonical kinematic traits and making it appear that lacks a classical stellar bulge when in fact its kinematics have been altered \parencite{saha12}. Parallel to the question of the existence of the classical bulge in the Milky Way is the study of the inner stellar halo, and particularly the degree to which it is an \textit{in-situ} component formed early in the Milky Way \parencite[e.g.][]{rix22}. These two putative stellar populations are linked by the similarities in their creation, both being old remnants of the earliest epochs of Milky Way formation. It may be the case that the classical stellar bulge is simply an inner extension of the \textit{in-situ} halo \parencite{perez-villegas17}. It is for these reasons that the study of the bulge continues to be an important pillar in the broader understanding of the history of the Milky Way, as well as the specific understanding of the stellar halo.


\subsection{The stellar halo of the Milky Way}

% Introduction to the stellar halo and why it is important
The final component of the Milky Way to consider is the stellar halo, which is deserving of its own section in light of its centrality to this thesis. Here will be presented an overview of the fundamentals of the stellar halo, focusing specifically on theory and accumulated knowledge prior to the launch of the \textit{Gaia} satellite. Much of the work presented here lays a superb foundation for more recent advancements enabled by \textit{Gaia}. The stellar halo is an approximately spheroidal assortment of stars embedded in and surrounding the Milky Way galaxy. Being completely devoid of star-forming gas, the stars in the stellar halo are characteristically old. They tend to be metal poor, consistent with their old ages, and typically move at high velocities with isotropic kinematics. The stellar halo gained promininance with the work of \textcite{eggen62}, who based their hypothesis of galaxy formation via dissipational collapse on the measured properties of fast-moving halo stars. \textcite{searle78} also based their arguments for galaxy assembly via hierarchical merging of smaller protogalactic fragments on the distribution and abundances of a prominant sub-population of the stellar halo: globular clusters. 

% Importance to galactic archaeology
The stellar halo is of principal importance to the endeavour of galactic archaeology for many reasons, and chief among them that it is the location in the Milky Way where dynamical timescales are longest. In other parts of the Galaxy stellar populations, such as the bulge or disk, mix together with one another, are subject to potentially chaotic perturbations from other components, or undergo complicated secular or externally driven evolution. While it is still possible to study these other stellar populations in the context of galaxy formation owing to their additional traits such as ages and chemical abundances, information can still be lost as stellar populations mix, and dynamical information about the formation of the Galaxy is undoubtedly erased. Additionally, any inferences are typically gleaned through the hazy lens of N-body or similar numerical models. This all contrasts with the stellar halo, where long dyamical timescales allow for stellar structures to maintain coherence over many billions of years. Stellar populations in the halo can still be confused and information lost via the same mechanisms as for inner-Galaxy populations, but in general at a much slower rate. The bottom line is that the stellar halo retains more valuable information from the earliest epochs of the Milky Way than other stellar populations, and so any effort to study the formation and evolution of the Milky Way should focus on the stellar halo.

% The dual stellar halo
Central to the study of the stellar halo is the notion that it is fundamentally of a ``dual'' nature \parencite{norris94,zolotov09}. It should contain two populations (in not necessarily equal amounts) of stars: one \textit{in-situ} population which was formed within the Galaxy, likely in its earliest epoch; and one \textit{accreted} population, formed in other smaller galaxies which have merged with and deposited their stellar contents in the halo of the Milky Way. Indeed, pre-\textit{Gaia} observations of the Milky Way stellar halo tended to reveal dualities in abundances, density, and kinematics \parencite[e.g.][]{kinman94,carollo07,nissen10,deason11}. As shall become evident when it comes to discussing more recent \textit{Gaia} findings, the accreted stellar halo is on solid footing while a picture of the \textit{in-situ} halo is only beginning to emerge and its true nature is somewhat controversial. Prevailing theories for the creation of the \textit{in-situ} halo include the remnants of the initial dissipational collapse and assembly of the proto-galaxy, the formation of stars from gas while it is being accreted (from the broader cosmic reservoir or stripped from an infalling satellite), and the disruption and heating of an early proto-disk \parencite{zolotov09,purcell10,font11,tissera13,cooper15}. One factor which complicates this simple dichotomy is that the very early Milky Way was likely built in large part via the merging of smaller proto-galactic fragments \parencite[i.e. within the standard galaxy formation frameworks of ][]{white78,white91}. So should these earliest stars be considered \textit{in-situ} or accreted? In practice recent works have distinguished these earliest stars as the ``proto-galaxy'' or ``young galaxy'' \parencite[e.g.][]{conroy22,belokurov22}. But nonetheless, the fact that it has become well-established that there are two separate formation mechanisms for distinct parts of the halo offers a valuable opportunity: study the accreted halo to learn about the impact of mergers, a primary growth-mode for spiral galaxies; and study the \textit{in-situ} halo to learn about the early conditions within the Milky Way itself.

% Differences in the properties of the dual halo
The accreted and \textit{in-situ} halo populations should differ in terms their abundances, kinematics, and densities, and within each population can exist trends based on the specific epoch and mechanism by which the stars were desposited into the halo. With regards to abundances, stars formed within the Milky Way benefited from its deep potential well, allowing the retention of much more enriched gas which leads to higher metallicities in formed stars, while accreted stars formed within the shallower potential wells of dwarf galaxies and globular clusters, which do not facillitate as rapid chemical enrichment for stars of a given age. This should lead to notable variations in [Fe/H], [$\alpha$/Fe], [Al/Fe], and [Mn/Fe] abundances \parencite{tumlinson10,zolotov10,hawkins15}. In terms of density profiles, the \textit{in-situ} halo is expected to be more centrally concentrated in the galaxy, and may overlap with the bulge region. But the specific location of these oldest stars formed within the Galaxy will reflect the early conditions of the Milky Way \parencite[e.g.][]{el-badry18}. Accreted stars, on the other hand, should be distributed in accordance with the geometry of the merger event that deposited them in the stellar halo, and are expected to occupy predominantly the outermost parts \parencite{abadi06}. Finally, the kinematics of the \textit{in-situ} population may be expected to again reflect the specific formation mechanism for this component. Accreted populations also inherit their kinematics from the merger event that deposited them in the stellar halo \parencite{bullock05,johnston08,cooper10}, and the retention of this information offers a powerful tool to reconstruct these mergers today. In aggregate, the accreted component should be roughly isotropic, as it is built of many mergers with assumed random accretion trajectories, and the kinematics should have a slight radial bias, reflecting the influence of dynamical friction which causes orbits of merging satellites to radialize.

% The accreted stellar halo as the potentially dominant component
Within the framework of $\Lambda$CDM and a Universe in which structure is hierarchical it is quite natural to assume that the stellar halo should be composed to a substantial degree of the remnants of mergers. Evidence from both cosmological and tailored N-body simulations suggests that the Milky Way should have experienced mergers with at least dozens, and up to 100-200 smaller satellites \parencite{abadi06,fakhouri10,font11,pillepich14}. Consistent with the halo mass function, a few of these satellites should be quite large (and therefore of higher metallicity) with stellar masses M$_{\star} \sim 10^{7}-10^{9}$~\Msun, and the majority should be smaller. Theoretical studies typically find that these few largest satellites dominate the resulting mass budget of the accreted stellar halo \parencite{bullock05,delucia08,cooper10}. Since dynamical friction acts more aggressively on more massive satellites, the debris from these few large accretion events should be concentrated in the inner halo. This prediction has been born out by findings from \textit{Gaia} that indeed show the inner stellar halo to composed of debris from a single major merger.

% Using kinematics to trace the individual mergers
The most interesting property of the merger remnants that constitute the accreted halo is that the kinematics of the debris from each individual event are approximately conserved (e.g. energy, angular momentum, actions) and reflect the circumstances of the merger \parencite{helmi00,johnston08}. This means that it is possible not only to glean the origin of this component of the stellar halo, but ideally to be able to compile a record of the individual mergers. This includes their masses, the time the merger occurred, and other relevant information about the constituent gas and stellar content of the accreted system. This is a very powerful prospect, since as explained in the previous section mergers have been linked to a myriad of formation and evolutionary mechanisms for many other components of the Milky Way, and by generalization other disk galaxies, and so it is worthwhile to characterize them. A good example of such an exercise in practice using pre-\textit{Gaia} data is the discovery of the ``Helmi streams'' by identifying substructure in orbital action space \parencite{helmi99}. The progenitor of the Helmi streams is thought to be a lower mass dwarf galaxy, perhaps akin to the extant Fornax dwarf spheroidal in the Milky Way stellar halo. As this exercise demonstrates, and as will be expounded in the next section, this catalogueing of mergers is not only possible but actively occurring in the \textit{Gaia} era.

% Sagittarius and other coherent accreted structures
So far the emphasis has been on ancient accretion events in the Milky Way. But testament to the recent and ongoing process of accretion can be seen now in observations of the stellar halo \parencite[e.g. the ``field of streams''][]{belokurov06}. A particularly significant example is the discovery and subsequent extensive study of the currently-merging Sagittarius dwarf galaxy \parencite[Sgr][]{ibata94}, the core of which is currently hidden behind the Galactic bulge. Sgr is on a polar orbit, and the tidal tails reflecting its doomed fate to be accreted into the Milky Way are evident above and below the disk \parencite{majewski03,belokurov06}. The merging of this dwarf is thought to contribute to the perturbation of the Galactic disk revealed by \textit{Gaia} \parencite[see the previous section][]{antoja18,laporte19}, and reinforces the important link between the evolution of the Milky Way and accretion events past and present.

% Other stellar halos 
Observations of the stellar halos of other nearby spiral galaxies, and particularly M31 (Andromeda), bolsters the notion that accretion is an ongoing process \parencite{mcconnachie09,martinez-delgado10}. The advantage of observing the stellar halos of other galaxies is that their surface brightness is much higher, as unlike the stellar halo of the Milky Way they are confined to a small solid angle and not spread over the whole sky. Such observations reveal the complexities of unmixed debris from recent accretion events: smaller disrupting systems produce arcs and tails during tidal stripping, while larger accretion events produce caustics, plumes, and shells \parencite[][provides a good overview of such phenomena]{johnston08}. The tidal tails emanating from stripped systems in the Milky Way are readily observable in the Milky Way due to their definiteness \parencite[e.g.][]{belokurov06}. But more diffuse caustics, plumes, shells, and other features are not easily inferred in the Milky Way and so observations of other galaxies coupled with N-body studies form the basis for our understanding of them. These are crucial to understand though, as old major mergers in the Milky Way will have undoubtedly left such complex signatures in the stellar halo, complicating the task of characterizing the properties and circumstances of the accretion event.

% Globular clusters
Up to this point the focus has been on the use of field stars to study the stellar halo, but a prominant sub-population of halo denizens are globular clusters \parencite[for catalogues see][]{harris96,baumgardt18}. Enigmatic objects, globular clusters are typically dense and bright (hence our extensive knowledge of them), but also very old and metal poor. Their ages are comparable to the age of the Universe, and they are thought to inhabit most galaxies. While their formation mechanisms are still not well understood \parencite{forbes18a}, they exhibit a number of useful internal properties such as having well-defined stellar populations, abundances, and kinematics \parencite[although many also have multiple populations with slight differences][]{milone22}. They also seem to obey broader population-level trends, such as having reasonably well-defined age-metallicity relations \parencite{forbes10,leaman13} corresponding to leaky-box chemical evolution models, or the fact that the number of clusters in a galaxy is correlated with its mass \parencite{harris13,forbes18b}. Finally, globular clusters belonging to dwarf spheroidals are largely expected to survive the process of their host merging with the Milky Way, owing to their dense nature \parencite{penarrubia09}. Indeed, examining the age-metallicity-kinematic trends of known globular clusters in the Milky Way reveals two populations: one with lower metallicities and extended halo-like kinematics, and one with higher metallicities and more concentrated disk-like kinematics. These two populations of GCs echo the \textit{in-situ} and accreted stellar halo, and indeed are expected to be of comparable nature.

% On the use of globular clusters
Despite a lack of satisfactory models that describe the formation of globular clusters, their empirical properties and the observed trends that they appear to hold to mean they are useful for galactic archaeology. Indeed, their ability to trace and describe accretion events has been demonstrated in the context of the merging Sgr dwarf spheroidal \parencite{law10}. In general, the kinematics of globular clusters should reflect the merger event, as for stars, and therefore accreted clusters should be roughly isotropically distributed with a range of kinematic properties \parencite[linking back to the hypothesis of ][]{searle78}. As with many of the other aspects of the stellar halo mentioned above, the \textit{Gaia} era substantially increases the amount of data with which they may be studied, and they will be used both to discover and characterize new and existing substructure in the stellar halo.

% Concluding paragraph
In this section we have seen how the stellar halo acts as a testing ground for many of the hypotheses of hierarchical galaxy formation in $\Lambda$CDM introduced in the first section of this introduction. Not only have we explored the theoretical links between the halo and galaxy formation theory, but directly touched on multiple aspects of the Milky Way today that can likely trace their origin, at least in part, to a merger or accretion event. These include the thick disk and the Galactic bar. Even before \textit{Gaia} data became available much progress had clearly been made, especially on the theoretical front. In the next section we will enter the \textit{Gaia} era and introduce many of the discoveries and advancements that it has facillitated.