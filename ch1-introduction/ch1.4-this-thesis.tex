\section{Thesis outline}

Example of multiple parenthetical citations 

% (\textcite{lane22}; first text block; \textcite{lane22}; second text block; \textcite{lane22})

% (\citeauthor{lane22}; \citeyear{lane22}, first text block; \textcite{lane22} second text block)

\subsection{Conventions used throughout the thesis}

% Choice of coordinate frame convention
Here I outline a number of conventions employed throughout the thesis. First, I generally work in a left-handed coordinate system, in line with the fact that the Milky Way rotates clockwise from the perspective of the Galactic north pole. While this is a common approach in Milky Way science others take a different approach, given that right-handed coordinates are almost guaranteed to be used in simulations for example, and obviously noting that the choice of Galactic north pole is arbitrary as it simply lies in the Earth's northern hemisphere. The choice of left-handed coordinates though does have the aesthetic benefit that the disk has positive angular momentum.

% Solar kinematic parameters
When considering observational data I make the following assumptions about the motion and location of the Sun with respect to the Galactic center. For the work presented in \james{Chapter 2 (link this)} I assume the distance is 8.178~kpc \parencite{gravity19}, and for the work on \james{Chapter 3 (link this)} I assume the distance is 8.275~kpc \parencite{gravity21}. These different choices reflect the prevailing reference at the time that the work in these chapters was done, and the work in \james{Chapter 4 (link this)} does not require knowledge of the distance to the center of the Galaxy. Regarding solar motion, I assume that the circular velocity at the location of the Sun is 220~km~s$^{-1}$ and that the velocity of the Sun with respect to the local standard of rest is $(U,V,W) = (11.1,12.24,7.25)$~km~s$^{-1}$ \parencite{schoenrich10}. Finally, when required I take the vertical height of the Sun above the Galactic disk to be 20.8~pc \parencite{bennett19}.

\section{Summary of chapters}

This thesis is outlined as follows. In Chapter 2 I present a analysis of the Milky Way stellar halo using nominal DFs and APOGEE data. The goal of this chapter is to investigate the completeness and purity of kinematically selected samples of stars belonging to the a DF representing the GS/E remnant. This Chapter is replicated from the published work of \textcite{lane20}. In Chapter 3 I employ the findings and lessons expounded in Chapter 2 to the study of the GS/E remnant using APOGEE data. A high-purity sample of GS/E stars is selected and to the sample I fit a density profile and derive the mass of the remnant. The work in this Chapter is taken directly from the published work of \textcite{lane22}. In Chapter 4 I present an analysis of the remnants of major mergers around Milky Way analogs found in the Illustris-TNG simulation suite. The findings of this Chapter, which have not yet been published, will hopefully provide valuable insights for future efforts to model merger remnants in our own stellar halo.
