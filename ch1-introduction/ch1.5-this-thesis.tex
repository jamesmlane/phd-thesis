\section{Thesis outline}

% (\textcite{lane22}; first text block; \textcite{lane22}; second text block; \textcite{lane22})

% (\citeauthor{lane22}; \citeyear{lane22}, first text block; \textcite{lane22} second text block)

\subsection{Conventions used throughout the thesis}

% Choice of coordinate frame convention
Here are outline a number of conventions employed throughout the thesis. A left-handed coordinate system is used when working with Milky Way observational data, in line with the fact that the Milky Way rotates clockwise from the perspective of the Galactic north pole. While this is a common approach in Milky Way science other authors sometimes take a different approach, given that right-handed coordinates are almost guaranteed to be used in simulations for example, and obviously noting that the choice of Galactic north pole is arbitrary. The choice of left-handed coordinates though does have the aesthetic benefit that the disk has positive angular momentum.

% Solar kinematic parameters
When considering observational data the following assumptions are made about the motion and location of the Sun with respect to the Galactic center. For the work presented in Chapter 2 the distance is 8.178~kpc \parencite{gravity19}, and for the work on Chapter 3 the distance is 8.275~kpc \parencite{gravity21}. These different choices reflect the prevailing reference at the time that the work in these chapters was done, and the work in Chapter 4 does not require knowledge of the distance to the center of the Galaxy. Regarding solar motion, it is assumed that the circular velocity at the location of the Sun is 220~km~s$^{-1}$ and that the velocity of the Sun with respect to the local standard of rest is $(U,V,W) = (11.1,12.24,7.25)$~km~s$^{-1}$ \parencite{schoenrich10}. Finally, when required, the vertical height of the Sun above the Galactic disk is assumed to be 20.8~pc \parencite{bennett19}.

% Consistency
With regards to abbreviations and language every effort is made to present consistency across thesis chapters, but they have been written at different times when different nomenclature has been in use (for example GS/E is referred to as GE in Chapter 2). Definitions are all ensured to be self-consistent within each chapter, and will be explained where necessary. While modifying the Chapters was an option, it was elected to keep them in their as-published form.

\subsection{Summary of chapters}

This thesis is outlined as follows. Chapter 2 presents an analysis of the Milky Way stellar halo using nominal DFs and APOGEE data. The goal of this chapter is to investigate the completeness and purity of kinematically selected samples of stars belonging to a DF representing the GS/E remnant. This Chapter is replicated from the published work of \textcite{lane22}. Chapter 3 employs the findings and lessons of Chapter 2 to the study of the GS/E remnant using APOGEE data. A high-purity sample of GS/E stars is selected and to the sample is fit a density profile and the mass of the remnant is calculated. The work in this Chapter is taken directly from the published work of \textcite{lane23}. Chapter 4 presents an as-yet unpublished analysis of the remnants of major mergers around Milky Way analogs found in the Illustris-TNG simulation suite. The goal of this work is to test whether or not standard DF models actually provide satisfactory fits to simulated remnants. The findings of this Chapter will provide valuable insights for future efforts to model merger remnants in our own stellar halo. Chapter 5 concludes the thesis with a discussion of the results, a summary of their impact on the field, as well as a look ahead towards future opportunities to use DFs to study the stellar halo well into the \textit{Gaia} era.
