\section{Summary of major results}

% True overview of results
This thesis has focused on the application of distribution functions (DFs) to the modelling of the stellar halo of the Milky Way in the \textit{Gaia} era. This has been approached from multiple perspectives, including theoretical work with N-body simulations, and observational work both on the stellar halo broadly, and the accretion remnant GS/E specifically. Here we briefly summarize the main results from each chapter before we move on to the discussion.

% Major results for chapter 2
Chapter 2 began with a look at the Milky Way stellar halo as a whole and the ways in which DFs might be applied to its study. We constructed a DF-based representation of the stellar halo as observed by APOGEE~DR16 and \textit{Gaia}, which included high and low anisotropy components representing GS/E and the remainder of the stellar halo respectively. With this model the efficacy of various kinematic spaces for the purpose of selecting GS/E stars was examined. This was done by computing purity and completeness across each kinematic space and identifying regions where the purity of the model GS/E sample was high, while trying to also maximize completeness. We find that the action diamond as well as eccentricity and angular momentum were the spaces that exhibited superior performance in their ability to generate high-fidelity GS/E samples. The purity and completeness of a range of other selection criteria used in the literature for GS/E was also reported. We found that many of these literature criteria were heavily contaminated by non-GS/E parts of the stellar halo in our model, and emphasized that authors need to account for thes effects when attempting to model GS/E using kinematically-selected samples. A byproduct of this analysis was the realization that stellar samples with complex spatial selection functions, such as APOGEE~DR16, can exhibit complex features in kinematic spaces that use energy. This has implications for the ongoing search for additional substructure in the stellar halo, and in particular may play a part in explaining the nature of the Heracles remnant. 

% Major results for chapter 3
In Chapter 3 we build on the results of the first project by studying the mass and density profile of the GS/E remnant. Kinematic selection functions were constructed for a variety of kinematic spaces using a modified version of the technique used in Chapter 2. With these selection functions, and the associated kinematic selection criteria, high purity samples of GS/E red giant stars are chosen in APOGEE~DR16 data. Density profiles are fit to these samples using a method that combines the kinematic selection functions with effective selection functions that account for the observational biases in the APOGEE data. The mass of GS/E is computed by integrating over the resulting density profiles and normalizing their amplitude using isochrones to account for the narrow slice of the red giant branch selected for modelling. We find that GS/E is represented by a triaxial ellipsoid with principal axes aligned towards the Hercules-Aquila Cloud and Virgo Overdensity, confirming previous results. The density profile is a power law with index of $\sim -2$ and exponentially truncated at between $15-25$~kpc. The mass of GS/E is about $1.5\times10^{8}$~\Msun, which is much lower than many other studies have estimated. Given standard stellar mass-halo mass relations and assuming the redshift of accretion, we estimate that GS/E constituted a minor 1:8 mass ratio merger, and that it currently comprises less than 25~per~cent of the stellar halo.

% Major results for chapter 4
Chapter 4 uses the IllustrisTNG simulations to study how well commonly used DFs, including those used in the first two projects presented in this thesis, actually correspond to the stellar phase space distributions of accretion remnants. We select 30 Milky Way analogs by stellar mass and disk-to-spheroid fraction criteria. Within these analogs, 116 significant (stellar mass ratio $>$ 1:20 at the time of accretion) mergers are identified and the remnants extracted at $z=0$. To these remnants are fit density profiles and three commonly used DFs: the constant anisotropy model (which is used in projects 1 and 2), the Osipkov-Merritt model, and a linear combination of Osipkov-Merritt models. We confirm that each accretion remnant approximately satisfies the Jeans equation, making them suitable for equilibrium modelling. We assess the performance of each DF in a self-consistent manner by comparing the velocity dispersion profiles and likelihoods of the data with samples drawn from each best-fitting DF. We also examine two case studies corresponding to a GS/E-like merger and a Sequoia-like merger. All of the DFs studied are able to satisfactorily capture the broad trends in velocity dispersion, energy, and angular momentum for most remnants, and only the specifics of the distributions and trends are sometimes lacking. Our findings indicate that Osipkov-Merritt DFs are superior at describing anisotropic remnants when compared to the constant anisotropy DF, which has implications for future GS/E modelling efforts. We also present a compendium of best-fitting merger remnant properties alongside commentary on them.