\section{Concluding points and future outlook}

% Starting paragraph
The stellar halo hosts extensive information about the formation and evolution of the Milky Way. The \textit{Gaia Space Telescope} and accompanying data, building upon many decades of theoretical underpinnings, set the stage for the most auspicious time in history to further our understanding of our home Galaxy. A major discovery of this era is that the Milky Way was subject to a substantial merger event early in its life, probably $10-12$~Gyr ago. GS/E, as it is known, has enraptured the astronomical community since its detection, and remains a topic of intense investigation. Alongside GS/E, a multitude of unique stellar halo populations have been recognized, some appearing to be \textit{in-situ} structures, born within the Milky Way at some early time, while others may have been accreted in disruptive interactions between the Milky Way and an incoming smaller galaxy. The task set before the modern astronomer is twofold: first, provide an accounting of each of these structures and their likely origin; and second, tie each structure into the complex landscape of the early Galaxy to synthesize a clearer picture of its genesis and first moments. DFs offer an important framework to aid in both of these pursuits, as they are the natural, physically motivated models for both the density and kinematics of such stellar populations.

% Describe the successes of this thesis
This thesis has used DFs to probe the Milky Way stellar halo, making a number of important advancements. Three projects have been completed, each touching on a different aspect of their use. In the first project DFs were used to illustrate how stellar populations mix and overlap in a variety of commonly used kinematic spaces. The purity and completeness of a number of well-defined selections was reported, and additional insights into the dangers of naive study of the stellar halo with samples that bear complicated selection effects was expounded. The second project focused specifically on the GS/E remnant, using DFs to enable the first accurate determination of its density profile and mass. It was found that GS/E was likely far less massive than previously reported, a finding with implications for its interpretation as an agent of change in the early Milky Way. Finally, a third project examined merger remnants in simulated Milky Way analogs, applying a variety of DF models to the data in an effort to determine their applicability. It was found that for anisotropic remnants resembling GS/E, that the Osipkov-Merritt DF is the most reliable model, and should be considered moving forwards.

% List of main takeaways
The principal message of this thesis can be summarized as follows: when studying the stellar halo in the \textit{Gaia} era it is important to slow down, consider the data thoughtfully, and rely on physically motivated models like DFs as the basis for inference. The takeaways of this body of research are as follows:

\begin{itemize}
    \item Stellar populations, both of accreted and \textit{in-situ} origin, have complex phase space distributions that overlap with one another. The selection of stars from one population may be done from a characteristic region of phase space, using a variety of kinematic properties, but must be done carefully, using DFs for example to inform about selection biases and completeness.
    \item Survey selection effects can cause the appearance of substructure when projected into a many kinematic spaces, and especially energy. Indeed this may impact the discovery of the purported Heracles merger remnant, which manifests as such an overdensity. The effects of survey selection must be considered when searching for new stellar halo populations, and investigating potential links between extant structures. DFs provide a natural framework to guide such work.
    \item The GS/E remnant can be described by a triaxial ellipsoidal density profile, which is shallow in the inner Galaxy, and steep in the outer Galaxy, breaking between $15-25$~kpc. The density profile is rotated in the Galactic plane and inclined with respect to it, aligning with several notable stellar halo overdensities.
    \item The GS/E progenitor was probably much smaller in (stellar) mass than previously suspected, being likely only a few times $10^{8}$~\Msun. The implications of this new reality on the specific scenario of its accretion must be investigated. All the behaviours in the Milky Way attested to it, such as the formation of the thick disk and Splash populations, must be scrutinized in light of its relegation from major to (still substantial) minor merger.
    \item Simulated accretion remnants in IllustrisTNG are well-fit by canonical DFs described by energy and angular momentum. It is found that Osipkov-Merritt models, and specifically linear combinations of them, provide the most satisfactory fit to anisotropic merger remnants like GS/E. Moving forwards others should consider using these models when studying GS/E.
\end{itemize}

% Open questions remaining about GS/E
Looking ahead, there are a number of important questions to consider in regard to the material studied in this thesis. First, in regards to GS/E, despite the major steps that have been taken to understand its origin, there is still much to learn. First, what are the implications for a lightweight GS/E? Can the behaviour attributed to GS/E in the early Milky Way survive its relegation from major to minor merger? Does the shape of GS/E change with radius? Can we validate the claimed epoch of the merger itself by examining the debris and gain a deeper understanding of the geometry and pace of the accretion event that created GS/E? Does the GS/E fit well within the $\Lambda$CDM context that we assume the Milky Way to exist in? Given the demonstrated successes in this thesis, DFs should have a central role to play in the answering of these questions.

% Study of other remnants and stellar populations
Beyond GS/E, a myriad of other stellar populations in the Milky Way have been discovered and are in need of characterization. Much in the same way that the nature of GS/E was clarified by careful investigation after its discovery, so will be fate of these remnants as well. For example, what is the reality of the Koala/Kraken/Heracles inner-Galaxy stellar halo population that has been claimed? Many smaller accretion remnants too have been discovered and need thorough study, including Thamnos, Sequoia, and the Helmi Streams among others. There are also \textit{in-situ} components of the Milky Way, such as Splash populations, or the newly discovered Aurora-type populations, which also require study. Such populations likely reveal the earliest moments of the Milky Way, and a picture of the transition from the first moments of Galactic formation to coherent rotation is emerging.

% Concluding paragraph
Before concluding, it is worthwhile to step back and consider what momentous change has unfolded in our understanding of the Galaxy during the span of this thesis. It began shortly after the discovery of GS/E, and ends with the community on the verge of probing the very formation of the Galaxy. The case here has been made for distribution functions to be a bedrock element of such studies. Indeed, the importance of their application has been demonstrated clearly in the changing of our conceptualization of the GS/E remnant. Their continued use is advocated for and this thesis has elucidated many of their potential uses and benefits as we continue to delve into the history of the formation and evolution of the Milky Way, in the \textit{Gaia} era.