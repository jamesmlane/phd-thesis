\section{Discussion of results}

% Introduction to the discussion
This thesis represents an important step forward in the understanding of the use of DFs for the study of the stellar halo in the \textit{Gaia} era. Much of the work done here is either at odds with prevailing opinions and practices in the community of Milky Way stellar halo studies, or challenges other findings. Most notably are the questions of the stellar mass of GS/E and the manner in which one selects GS/E stars for modelling. Additionally, it is worthwhile revisiting some of the results from the first two projects in this thesis in light of the results from the third project on the use of DFs to study simulated accretion remnants. Discussed here are some of these points, as well as the other results from this thesis synthesized together.

% On the selection of GS/E stars
\subsection{The selection of GS/E and other stellar samples}

The first project in this thesis largely concerned itself with the selection of GS/E stars for the purpose of modelling. This is important given that most papers studying GS/E observationally perform some sort of selection, usually kinematic, to define the sample. Questions arise though when this is done. How pure and complete is the resulting sample? How can the (im)purity of such selections impact the science questions under consideration? Our conclusions were first and foremost that, if not careful, it is easy to achieve sample purity of only $50-70$~per~cent when using broad selections. We concluded that work with a series of recommendations for other authors.

% Carrillo+ 2024
An interesting work that followed these results was presented by \textcite{carrillo23}. These authors study kinematic selections for GS/E using simulations, as opposed to our data driven approach. They draw similar conclusions as us, that many simple kinematic cuts commonly used to define GS/E can substantially bias subsequent inference. Notably they also criticize the commonly used definition for GS/E of eccentricity~$> 0.7$ which they claim leads to an overestimation of resulting masses, a conclusion we draw as well. The cause of this is that a large fraction of the non-GS/E stellar halo will incidentally have eccentricity above this value, overinflating the size of any sample thus selected. Such biases help to explain why some estimates for the mass of GS/E tend to be quite large \parencite[e.g.][, who use this eccentricity cut to define GS/E]{han22}. \textcite{carrillo23} also raise the point that it may be more advantageous to employ a chemical selection for GS/E depending on the science goal. The problem is that typically the science goal revolves around deciphering the chemistry of GS/E so as to learn more about the properties of the progenitor, making it inappropriate to select it via abundances, and instead to rely on its characteristic kinematics.

% The impact: kinematics and density
Beyond the stellar mass bias there are additional unexpected effects stemming from the choice of kinematic selection. First, inferences regarding the shape and extent of the density profile may also be misleading, since a large fraction of the stars in the sample will trace the density profile of the low-anisotropy Milky Way (undoubtedly a mixture of \textit{in-situ} and other accretion remnants). Such effects could help to explain the differences between the finding, shared by us as well as \textcite{iorio21}, that the principal axis of the GS/E density profile rotated about $90^\circ$ from the Galactic center-Sun line and inclined above the disk, versus those of \textcite{han22} who find a very different orientation. Although differences in the effective volume of those samples (i.e. the Galactocentric radii probed by the tracer stars) could reflect real differences in the geometry of GS/E \parencite[see][ for an example of this in the outer halo]{chandra23}. 

% The impact of selection effects
On a similar note, selection effects such as those noted in Chapter 2 could seriously complicate studies of GS/E (the example of overdensities in energy-angular momentum space was used). This is especially pertinent when considering that the GS/E progenitor likely had an internal metallicity gradient which would be imprinted also on the remnant. This occurs because when GS/E was accreted the outer parts were likely stripped first and deposited in the outer parts of the Milky Way stellar halo, while the inner parts would get dragged into the inner Milky Way by dynamical friction \parencite[e.g. see ][]{amarante22,vasiliev22}. Therefore observational data with poorly constrained selection effects will suffer from a clouded view of what is likely a complicated chemical reality of the GS/E remnant. Another remnant that is potentially caught up in this sort of issue is Heracles \parencite{horta21a}. In Chapter 2 it is demonstrated that selection effects in APOGEE~DR16 can lead to the appearance of substructure in energy space, arising from a multi-modal distribution in Galactocentric radius of the APOGEE data. This raises the question: is Heracles actually a distinct merger event? \textcite{amarante22} argues that Heracles cannot simply be an extension of GS/E, with a separation driven by the selection function, since at lower energies (where Heracles manifests) one would expect higher metallicity GS/E debris for the reasons noted above. On the other hand, it could simply be that a divot is taken out  of an otherwise smooth energy distribution of the inner stellar halo, comprising a combination of accretion remnants and old \textit{in-situ} stellar halo populations such as the Splash and Aurora (see below). Indeed, \textcite{horta23a} do find that Heracles is not dissimilar from \textit{in-situ} Milky Way stellar halo populations, perhaps suggesting this as the resolution to the puzzle.

% Modelling mixtures
In light of these complicating factors perhaps it is best to consider studying GS/E in the future using some sort of mixture model or perhaps a model including contamination. Early work in this spirit has already been done to some extent. For example \textcite{lancaster19} and \textcite{iorio21} study the kinematics of GS/E with Gaussian mixtures. The downside of their specific approach is that Gaussian mixtures do not necessarily correspond to physically-motivated DFs, and so we envision a future where mixtures of realistic DFs can be used to investigate GS/E and either simultaneously study the low-anisotropy stellar halo or marginalize over it. Much of this thesis has been written with an eye towards this future, and a number of barriers such as computability (to be touched on later) still lie in the way. This type of approach will be especially crucial when it comes to studying other remnants and stellar halo populations which are smaller in mass and therefore stand out less obviously in kinematic space and suffer from higher degrees of contamination. Work on GS/E has been feasible in this thesis in large part because of its dominance in the nearby stellar halo, which allows for a reasonable selection of its constituent stars for modelling with an assumed DF.


% The stellar mass of GS/E
\subsection{The stellar mass of GS/E, was it a major merger?}

% Introduction and stellar mass estimates
In Chapter 3 we report the mass of the GS/E remnant to be about $1.5\times10^{8}$~\Msun. Our result here follows a recent trend of lower mass estimates for GS/S. For example \textcite{han22} estimate about $5-7\times10^{8}$~\Msun. Interestingly, these authors define GS/E using the eccentricity~$> 0.7$ cut that both we and \textcite{carrillo23} criticize for being biased towards higher-mass selections. It is therefore likely the case that their mass estimates are biased to be slightly higher than expected. Indeed \textcite{carrillo23} also estimate the mass of GS/E using the selections they study as well as simulations and find $4\times10^{8}$~\Msun. \textcite{rey23} also estimate that GS/E constituted a lower mass ratio merger, between 1:8 to 1:20 which corresponds with roughly a few times $10^{8}$~\Msun. \textcite{callingham22} find a mass for GS/E of $3.2\times10^{8}$~\Msun\ using globular clusters. Finally, the earlier study of \textcite{mackereth20} has estimated a stellar mass for GS/E of $\sim 3\times10^{8}$~\Msun\ using density modelling and a selection in abundance. The conclusions here are clear: more recent estimates of the stellar mass of GS/E put the stellar mass at a few times $10^{8}$~\Msun\ and certainly $< 10^{9}$~\Msun. This is in stark contrast to early estimates which tended to estimate about $10^{9}$~\Msun\ or greater. These stellar mass estimates at the present day imply likely total halo masses at the epoch of accretion of about $10^{10.5-10.9}$~\Msun. Regarding the mass ratio of the merger, this indicates a much lower value of about 1:8-1:10 given the assumed mass for the Milky Way at this epoch (about half its present-day value, probably $\sim 10^{11.7}$~\Msun). This raises the question, could GS/E have actually been a minor merger?

% Why were mass estimates off?
One question that must be addressed is why were early mass estimates for GS/E higher than more recent estimates? Well, early stellar mass estimates for GS/E were driven by two types of study: those which examine its chemistry and attempt to derive the mass via fitting of chemical evolution models, and those which attempt to use simulations to identify GS/E analogs. In regard to chemistry, such analyses require many assumptions, and it is obviously possible that GS/E has a unique or complicated chemical evolutionary history \parencite[e.g. as argued by][]{matsuno21,sanders21}. Future studies on the chemistry of GS/E should focus on using its stellar mass as an anchor to understand its chemical evolution. 

In regard to simulations, most of such studies look to affiliate anisotropic, metal-rich debris in the stellar halo with those found in Milky Way analogs in simulations \parencite[e.g.][]{fattahi19,mackereth19a}. Such studies are not able to directly measure the mass of GS/E, but instead identify the types of mergers that lead to such remnants in a cosmological context. Interestingly, the new low stellar mass estimates are not at odds, per se, with the results of \textcite{fattahi19} or \textcite{mackereth19a}, but instead are simply on the low-end of estimates. \textcite{rey23} use the `genetic modification' technique \parencite{stopyra21,rey22} to study GS/E-type mergers of a range of mass ratios, from 1:2 to 1:24, in simulated Milky Way analogs. Puzzlingly, they find that high and low mass ratio mergers produce very similar remnants at $z=0$. These degeneracies are driven by differences in the way the merger plays out \parencite[see also][]{jean-baptiste17}. \textcite{grand20} and \textcite{orkney22} similarly find that multiple plausible accretion scenarios can all lead to similar distributions and amounts of GS/E debris in the stellar halo today. This all indicates that simulation-based inference of GS/E properties can be clouded by the complexities of the merger event, and accurate observational constraints should first and foremost inform our understanding of the fundamental properties of GS/E.

% Why are dominance assumptions off?
Another point to consider is how inspection of the GS/E density profile can help to explain why the remnant appears to be dominant in many stellar samples. These samples are typically concentrated near the position of the Sun, which happens to be close to the radial range where GS/E makes up the largest fraction of the stellar halo. The density profile of GS/E is broken at around $15-25$~kpc according to the work in this thesis as well as, for example, \textcite{han22}. At much larger distances the density of GS/E drops substantially compared with the rest of the halo. GS/E also has a particularly flat density profile in the inner Galaxy, and so its contribution will be lower there as well. Therefore when examining samples near the Sun, GS/E will appear to be the largest stellar population, but taken as a part of the whole stellar halo it contributes a smaller than expected, yet still significant fraction. This chance coincidence certainly bolstered the early idea that GS/E constituted a major merger experienced by the Milky Way.

% The major merger of GS/E
Given that GS/E may have been less massive than proposed soon after the time of its discovery, we must revisit many of the theoretical claims made on its behalf. It is obviously beyond the remit of this discussion---and indeed this whole thesis---to assess the sensitivity of each of the following scenarios to a variable GS/E progenitor mass. So just an overview of the main points in the context of a lighter GS/E mass will be given. First is the claim that the merging of the GS/E progenitor with the Milky Way led to the creation of the thick disk \parencite[e.g.][]{helmi18,gallart19}. This claim is also related to the idea that the metallicity of the Milky Way could have been `reset' in part by the arrival of a gas-rich GS/E in concert with cosmological gas accretion \parencite[e.g.][]{grand20,renaud21,ciuca24}, leading to the dual low- and high-$\alpha$ disks we see today \parencite[see also the two-infall model of e.g.][]{chiappini97}. The observational studies making these claims are largely based on an apparent agreement between the ages of the youngest stars in GS/E and the bulk of the stars in the thick disk. But obviously this match does not imply causation. In regard to the theory, it is convincing, yet there are many other mechanisms that have been proposed to create the thick disk. Indeed, the fact that GS/E may be of lower mass might actually be a boon for the formation of the thick disk. Since one of the favoured thick disk formation scenarios is early (i.e. before GS/E accretion) clumpy star formation, there must be a plausible path for such a hot disk to survive to the present day and a heavy GS/E could make that challenging \parencite[see e.g. arguments for this in][]{deason24}. 

Another, related question is whether the GS/E merger could have been responsible for creating the Galactic bar \parencite{fragkoudi20,merrow23}. The sensitivity of accretion-based bar formation channels to satellite mass is complex, yet the modern era is one in which the bar is receiving extensive study, and so a resolution to this question is likely forthcoming. The GS/E is also an important ingredient in the census of all Milky Way satellites, accreted and surviving. But work on integrating it into the broader $\Lambda$CDM context is still in the early, highly contested stages \parencite{fattahi20,naidu22}. Finally, the mass of GS/E is actually very important in the context of dark matter particle searches, since GS/E has contributed a not insignificant fraction of the dark matter in the Milky Way halo \parencite{necib19,evans19}. These searches rely on priors built from the assumed velocity distribution of dark matter in the halo, and the amount contributed by GS/E, which will be highly anisotropic like the remnant stars, is inferred based on the stellar mass of the remnant. Each of the propositions here summarized will obviously need to be revisited in light of a potentially low-mass GS/E, and all can certainly be studied with modern tailored and cosmological N-body simulations. 

\subsection{The DF of GS/E}

% Introduction to the question of DFs and the case for Osipkov-Merritt
A thread tying each of the three projects of this thesis together is the following question: what is the proper DF for GS/E? While in the first project we relied on the constant-anisotropy DF, driven by results at the time showing the large, effectively constant anisotropy of GS/E at all radii \parencite[e.g.][]{belokurov18,lancaster19}, recent observations and theoretical considerations do not support this. In Chapter 4 we found that Osipkov-Merritt DFs, specifically linear combinations of them, provide a much more satisfactory fit than constant-anisotropy DFs to the phase space distributions of GS/E-like remnants in IllustrisTNG. Such a finding is actually backed up by the work of \textcite{iorio21}, who trace the anisotropy profile of GS/E closer to the inner Galaxy than other studies (a result of the choice of RR Lyrae as a tracer) and find evidence for a drop in the anisotropy within 5~kpc. Such a scale radius for an Osipkov-Merritt profile (the radius at which $\beta=0.5$) is comparable to the typical scale radii we find for anisotropic remnants in IllustrisTNG. Finally, theoretical considerations also argue for the unsuitability of constant-anisotropy models with high $\beta$ on the grounds that they are actually unstable in their inner regions, and therefore should be thought to be unphysical \parencite[see][and references therein]{binney14d}.

% The effect of an Osipkov-Merritt DF on results
This is an important point for future modelling efforts, which should be encouraged to use Osipkov-Merritt models when studying GS/E, since they are not particularly more challenging to work with computationally. One important consequence of this, though, is the need to revisit the results of the first and second projects in this thesis in light of the realization that Osipkov-Merritt models are preferred. Since Osipkov-Merritt models lack extremely radial stars in the innermost part of the Galaxy they do manifest differences at the Solar position. In particular, they will end up predicting fewer stars on the most radial orbits (i.e. high eccentricity and radial action with low angular momentum) when compared with a similar constant-anisotropy model. This has implications for the results of the first two thesis projects. In the first thesis project we could expect to have lower completeness and purity in our extremely stringent selection criteria, which actually may pull our results in-line with the simulation-focused work of \textcite{carrillo23}. This would not likely have changed any conclusions, since all selection criteria we constructed, as well as literature criteria, were benchmarked with the same models.

% The effect of an Osipkov-Merritt DF on the stellar mass of GS/E
In the second thesis project on the stellar mass of GS/E the choice of constant-anisotropy model unfortunately plays a larger role. Since an Osipkov-Merritt model predicts fewer stars on extreme radial orbits, the kinematic selection functions used will end up having a smaller typical value. In other words the completeness of a given kinematic selection will be lower, and the effect of the kinematic selection function will therefore be to increase the resulting inferred mass, all else being the same. We have recently investigated the magnitude of these differences and find that the effect is likely about a factor of $1.5-2$ increase in the predicted mass of GS/E. Taking our best mass estimates for GS/E, which were about $1.5\times10^{8}$~\Msun, this would raise the mass to between $2.3-3\times10^{8}$~\Msun. This actually brings our estimate into much better agreement with other recent results that have estimated the mass of GS/E \parencite[e.g.][]{callingham22,han22,carrillo23} as well as some older results that predict a similar mass \parencite[e.g.][]{kruijssen19b,mackereth20}.

% Other DFs
In this thesis we have largely focused on DFs defined by energy and angular momentum, such as constant-anisotropy and the Osipkov-Merritt models. Other DFs for spherical stellar systems do exist, and their suitability for describing GS/E, as well as other populations in the stellar halo, should be considered. The most noteworthy alternative models are the action-based DFs \parencite{binney14d,posti15}, which were described the introduction to this thesis. These DFs are advantageous for a number of reasons. First, they are functions of the three actions, as opposed to just energy and angular momentum, allowing them to describe a greater range of kinematic populations. Second, they are easier to compute, not being defined by integral inversions in the way that the constant-anisotropy and Osipkov-Merritt DFs are. The downside of these models is that they require actions as inputs, and as was extensively discussed in the introductory section on actions, their computation requires approximations in any realistic situation. Energy and angular momentum, on the other hand, are well-defined quantities for any orbit. Nonetheless, action-based DFs are certainly worthy of investigation for use in modelling of the stellar halo.

% DFs including metallicity and age
Up to this point we have only considered DFs which are functions of kinematic parameters, and therefore position and velocity. But there are other properties of stars, such as abundances and ages, which can be just as characteristic, and are essentially invariable. It is therefore appropriate to consider the development of joint DFs that factor in kinematics, as well as abundances and ages. Work on this has already been done to some extent by \textcite{sanders15b}, who study `extended' distribution functions which combine metallicity and abundances. Such DFs will be invaluable for application to stellar halo populations, which also more often than not have unique abundances and chemical evolution histories. While it would be advantageous to factor in ages, obtaining them reliably for very old stars is still a challenge and so may not be prudent yet. But to include abundances into most DF-based frameworks, including those used in this thesis, should be simple since as a first step the metallicity DF can be thought to be independent of kinematic parameters. But more complicated DFs where the abundances are linked with various kinematic properties (e.g. to create a metallicity gradient) are also feasible to construct.

% Other options for DFs
One major barrier to the use of DFs based on the Eddington inversion, or similar models such as the Osipkov-Merritt and constant-anisotropy DFs, is computability. This issue was actually a major driver for our choice of fixing the DF in the first and second projects. Assuming that no strategy can be devised to drastically improve computability, the idea of trying to fit DFs to the stellar halo with variable parameters seems infeasible at this time. Certainly action DFs may prove to win out in the future for this reason alone. But other DFs and approaches based on machine learning may also be valid options. For example, \textcite{green23} present an approach to studying DFs using normalizing flows, which can learn the underlying DF distribution from a snapshot of phase space described by a tracer (i.e. how stellar halo stars trace an underlying distribution function). Such an approach may help with computability, since once trained such neural networks can make predictions quickly and efficiently. They could also help alleviate tensions between remnant distributions and DF models, such as those noted in Chapter 4, since normalizing flows can flexibly describe nearly any distribution. This approach may suffer, however, from the typical shortcomings of machine learning-based approaches: the result lacks any definite mathematical expression for the DF, and the `black box' nature of neural networks can raise skepticism. Another strategy that may bear fruit is the use of emulators based on neural networks to efficiently compute more standard DFs. Here a neural network would be trained to compute the value of the given type of DF given an input potential, tracer for the density profile, and DF parameters. The advantage of this approach would be that standard, physically-motivated DFs can be used, it is easily testable using existing numerical tools, and yet perhaps enables live fitting of DFs.

\subsection{The application of DFs to other stellar halo populations}

% Introduction and remnants
Many unique populations define the stellar halo in the \textit{Gaia} era, beyond GS/E which has been the emphasis thus far. As described in the introduction, there are many other supposed accretion remnants, including the Sequoia \parencite{myeong19}, Thamnos \parencite{koppelman19b}, and the Helmi streams \parencite{helmi99,koppelman19a}. In the inner Galaxy there is also possible evidence for a major merger more ancient than GS/E from stellar data, but more importantly globular cluster data. This is known as the Kraken \parencite{kruijssen20}, the Koala \parencite{forbes20}, or Heracles \parencite{horta21a}, although each of these structures may not all be aliases for the same population. DFs certainly have a role to play in characterizing each of these. As shown in Chapter 4 a wide variety of simulated merger remnants with a range of properties can be fit by simple DFs based on energy and angular momentum. A prescription for rotation can also be easily added, which will certainly be important for rotating stellar populations such as Sequoia. It is the study of these other remnants that is the main reason for wanting to establish a flexible framework of DF mixtures, since the study of any smaller remnant will be undoubtedly complicated by the more dominant stellar halo populations such as GS/E, but also stellar halo interlopers such as the thick disk and bulge.

% In-situ remnants
Another class of stellar halo population that has received increasing attention in recent years are those of an \textit{in-situ} nature. The most peculiar of these is the Splash population, discovered by \textcite{belokurov20}. The Splash population is thought to be a (partial) remnant of the early Milky Way disk that was energetically `splashed' up into the stellar halo after some merger, likely with the GS/E progenitor \parencite[although see][for a different hypothesis]{amarante20b}. Characterization of the Splash population, which is flattened and weakly rotating, would be straightforward with a DF-based framework. A major challenge, however, would be that the Splash is of lower present-day stellar mass than some other halo populations and therefore modelling it in a joint way with other DFs representing the major stellar halo populations, is probably important.

% The aurora
The most recently studied \textit{in-situ} stellar halo component is the Aurora population, discovered by \textcite{belokurov22} who examine the stellar halo in terms of [Al/Fe] abundance. Al yields are dependent on stellar metallicity and so [Al/Fe] is very sensitive to galaxy size and mass, since those drive metallicity enrichment generally. It is on this basis that \textcite{belokurov22} claim the discovery of Aurora, which has elevated [Al/Fe] abundances compared with GS/E, for example. Moreover, the Aurora population appears to increase in rotational velocity as metallicity increases in a manner that indicates the kinematic transition from a pressure-supported population to rotation-supported population, a process referred to as disk `spin-up'. \textcite{belokurov22} therefore claim that this population is the genuine \textit{in-situ} stellar halo, the original component of the Milky Way formed before the disk was established and before the GS/E merger. Building on this work, \textcite{conroy22} and \textcite{rix22} extend these findings to lower metallicities, confirming these findings and tracing the Aurora population to the innermost parts of the Galaxy. These findings and the notion of disk `spin-up' has been furthered by the study of globular clusters in the work of \textcite{belokurov23a} and \textcite{belokurov24}. Again, DFs, and especially joint metallicity-kinematic DFs, can certainly play a role in modelling these oldest, low metallicity, inner halo stellar populations.