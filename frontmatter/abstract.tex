\begin{abstract}
    Disk galaxies, such as our own Milky Way, populate the entire extent of the observable Universe and constitute its most fundamental building blocks. Among the many hallowed endeavours of modern astronomy is to cultivate a deep and hollistic understanding of the way these galaxies form, grow, and evolve. The stellar halo of the Milky Way hosts the fossil remnants of past accretion events, which trace one of the key growth modes of disk galaxies. Distribution functions are mathematical models for the phase space distribution of stellar populations, and may be used to study accretion remnants. This thesis presents work done to integrate the use of distribution functions into the study of the stellar halo and the accretion remnants which largely comprise it in the era of new data from the \textit{Gaia Space Telescope} and large ground-based spectroscopic surveys.
\end{abstract}