\begin{abstract}
    Disk galaxies, such as our own Milky Way, populate the entire extent of the observable Universe and constitute its most fundamental building blocks. Among the many hallowed endeavours of modern astronomy is the cultivation of a deep and holistic understanding of how these galaxies form, grow, and evolve. The stellar halo of the Milky Way hosts the fossil remnants of past accretion events, reflecting the habit of the Milky Way to devour smaller nearby galaxies over cosmic timescales. Such accretion episodes are among the most formative events experienced by the Milky Way, contributing both to its genesis at early times as well as subsequent growth and evolution. Among the most important accreted stellar population in the halo is \textit{Gaia}-Sausage/Enceladus, the remnant of a substantial ancient merger event. Distribution functions are mathematical models for the phase space distribution of stellar populations, and are useful tools for the study of stellar halo populations. This thesis demonstrates work done to integrate the use of distribution functions into the study of the stellar halo, and the accretion remnants which largely comprise it, in the era of new data from the \textit{Gaia Space Telescope} and large ground-based spectroscopic surveys.

    Three projects are presented which work towards the development of a better understanding regarding the use of distribution functions. First, we employ distribution functions to construct data-driven models of the Milky Way stellar halo and investigate their properties. Second, we use distribution functions to aid in fitting density profiles to the \textit{Gaia}-Sausage/Enceladus remnant; upon determining its mass, we find it to be far less than previously thought. Third, we assess the applicability of a wide range of distribution function models by testing them on simulated merger remnants in Milky Way analogs found in the IllustrisTNG simulation. Among our key findings are: that it is crucial to consider selection effects carefully when searching for stellar halo remnants and other populations; that the \textit{Gaia}-Sausage/Enceladus remnant was likely deposited during a minor merger; and that it is probably best modeled using combinations of Osipkov-Merritt distribution functions in future studies. The principal conclusion of this thesis is that distribution functions can, and should, play a central role in the modelling of the stellar halo, as we seek to learn more about the formation and evolution of the Milky Way in the \textit{Gaia} era. 
\end{abstract}